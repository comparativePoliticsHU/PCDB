\subsection{Type B Volatility in Lower House Vote Shares}\label{view_lhelc_volb_vts}
View \texttt{\footnotesize view\_lhelc\_volb\_sts} is based on tables Lower House, Lower House Vote Results and Party, and provides data at the level of lower house elections.

Type B volatility quantifies the change that occurs in the distribution of vote shares within parties in subseqent elections, comparing the results in the current election to that of the previous. 
Accordingly, type B volatility considers only so-called stable parties and measures the  volatility in the distribution of votes arising from gains and losses of these stable parties	

The formular to compute \texttt{\footnotesize lhelc\_volb\_vts} is
\begin{equation}
Seat\,B\,Volatility\,(k) = \frac{ | \sum\limits_{j=1}^{Stable} v_{j,(k-1)} - v_{j,k}| }{2}
\end{equation}, where $v$ are vote shares that party $j$ gained in the current lower house $k$ or in the previous lower house $k-1$.

View \texttt{\footnotesize view\_lhelc\_volb\_vts} is programmed as follows:

\lstinputlisting[%caption={Code to compute type B volatitlity ins lower house votes.},%
language=SQL]%
{../SQL-codes/view_lhelc_volb_vts.sql}

Stable parties are identified computationable by calculating the cross-product between rows in the subqueries 
\texttt{\footnotesize CUR\_LHELC\_VTS\_SHR} and \texttt{\footnotesize PREV\_LHELC\_VTS\_SHR}, and reporting only those for which a party identifier is enlisted in both the previous and the current election (cf. corresponding \texttt{\footnotesize WHERE}-clause). 

{\bf Note}: The concept of stable party makes no sense for first recorded lower house elections, and hence B volaities are not computed. 
The measure is highly sensitive to missing data, as no aggregate value is computed for lower house  elections in which at least one party except the group `Others withour seat' has NULL records for total vote results (cf. consistency check \texttt{\footnotesize cc\_missing\_lhelc\_pty\_sts\_records} [\ref{cc_missing_lhelc_pty_sts_records}]). A lack of reliable lower-level data thus causes severe lack of aggregate data. 

Generally, consistency check \texttt{\footnotesize cc\_lhelc\_volb\_vts} [\ref{cc_lhelc_volb_vts}]) provides for a comparison of the computed and the recorded figures, though the recorded have been computed manually as well.

