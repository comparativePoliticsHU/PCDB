\section{Keeping the PCDB updated}\label{sec_keeping_the_PCDB_updated}

Data in the PCDB is manipulated using \texttt{PostgreSQL}'s data manipulation language (DML) operations \texttt{INSERT}, \texttt{UPDATE}, and \texttt{DELETE}. \footnote{\url{https://www.postgresql.org/docs/9.3/static/dml.html}}

The following paragraphs will use the cabinet table (see subsection \ref{subsec_tab_cabinet}) in the \texttt{config\_data} schema of the \texttt{polconfdb} database as an example to introdcue some minimal working examples.

These examples can easily be applied to the other tables in the PCDB. 

\paragraph{Some words of caution} 
Please do not manipulate (i.e., insert, update, or delete) data without having a clear idea of 
\begin{itemize}
\item[a)]what is the primary key of a given table or the columns that uniquely identify rows;
\item[b)]which referential dependencies are implied by the structure of the PCDB; and accordingly,
\item[c)]how incomplete inserts or updates, or thoughtless delets affects the integrity and constistency of the PCDB.
\end{itemize}
Read about primary keys and the implementation of referential dependcies using foreing keys in the \texttt{PostgreSQL} documentation.\footnote{\url{https://www.postgresql.org/docs/9.3/static/ddl-constraints.html}}

With respect to the minium workin example, 
(a) The cabinet identifiers column (\texttt{cab\_id}) is primary key of cabinet table, and cabinet start date (\texttt{cab\_sdate}) in combination with the country identifier (\texttt{ctr\_id}) uniquely identify observations (i.e., rows). 

With reespect to (b), \texttt{cab\_id} is referenced as foreign key in the cabinet portfolios table (see subsection \ref{subsec_tab_cabinet_portfolios}), and, in combination with the party identifier \texttt{pty\_id}, uniquely identifies cabinet portfolios
Moreover, as cabinet compositions (i.e., rows in the cabinet tables) sequenced alongside lower house, upper house, and presidency configurations in the configuration events view, cabinet compositions are essential to compute configuration-specific indicators, such as cabinet parties cumulated seat share in the lower house; to identify open veto points; etc.
%Lastely, and \texttt{\footnotesize cab\_id}s are selected by several triggers to identify previous or subsequent cabinets for any given cabinet (subsections \ref{} and \ref{}).

Finally, in view of (c), though it is possible to insert a new observation to table Cabinet without providing, for instance, its start date, this would cause non-trivial problems, for instance, when compiling the configurations events view.

Users are thus strongly inclined to pay attention to the key and uniquness constraints of a given table when inserting, updating or deleting data from it. Information on constraints is provided in the respective subsections of the Table section (\ref{sec_table}) and the PCDB Codebook (see documentation Appendix).

\paragraph{Some words on data consistency}

Note that the trigger structure and functions defined on the \texttt{config\_data} schema ensures that manipulation executed on the cabinet, lower house, upper house, presidential election, and veto points tables propgate through to the configuration events and configuration country-years tables.
The interrelation between the configuration tables and the structure is explained in detail in sections \ref{sec_views_in_config_data_scheme}, \ref{sec_mviews_in_config_data_scheme} and \ref{sec_triggers}.

In other cases, such as the interrelation between the cabinet portfolios on the cabinet table, dependencies exist, but consistency is not enforced using a trigger structure. If you insert a new cabinet configuration, you have to manually add the corresponding cabinet portfolio (rows of parties in cabinet and the parliamentary opposition). No error will be raised if you fail to do so.
Likewise, if you record a new lower house election (upper house election), you have to make sure that the corresponding vote results are listed at the party level in the lower house vote results table, and that you record the lower house (upper house) configuration that corresponds to the election.
And if you record a new lower house (upper house) composition, you have to make sure that the corresponding seat results are listed at the party level in the lower house seat results (upper house seat results) table. 

  \subsection{Manually inserting data}\label{subsec_manually_inserting_data}

Adding a new row (i.e., an observation) to a table is proceeded with the \texttt{INSERT INTO}-command, by simply specifying the table (and schema), then the target columns, and third the values to insert.
Though insertation does not requiere to specify the target columns, as the original order of columns of a table is used as default, specifying target columns corresponding to insert values is best-practice, as it ensures a correct insert operation.

Here a minimum workin example:

\begin{lstlisting}[language=postgreSQL]
INSERT INTO config_data.cabinet
	(cab_id, ctr_id, cab_sdate, cab_hog_n, cab_care)
	VALUES (6038, 6, '2017-01-01', 'Licht', 'FALSE');
\end{lstlisting}

Note that the values you attempt to insert need to match the specified types of the target columns. 
If you attempt to insert a value that does not match the type of the respective column, an error message will be raised.\footnote{%
To recall the type of a given column, refer to the Codebook or browse the properties of the given table in \texttt{pgAdmin3} (left click on table in menu bar, and view `SQL pane').}
You can avoid such error messages, if you type instead 

\begin{lstlisting}[language=postgreSQL]
INSERT INTO config_data.cabinet
	(cab_id, ctr_id, cab_sdate, cab_hog_n, cab_care)
	VALUES (6038::NUMERIC(5,0), 6::SMALLINT, '2017-01-01'::DATE, 
    'Licht'::NAME, 'FALSE'::BOOLEAN);
\end{lstlisting}

Always refer to either the Codebook or browse the properties of the given table in \texttt{pgAdmin3} before you attempt to insert data into a table, as there exist constraints (e.g., \texttt{NOT NULL}, \texttt{PRIMARY KEY}, or \texttt{UNIQUE}) on some of the columns, which require inserting a value to these specific columns when adding a new row to the table.

Also, it is best-practice to assign ascending integer counters  to subsequent instituion configurations withn countries.
Finally, remember that the primary key of the cabinet table, \texttt{cab\_id}, contributes to the unique identification of observations in the cabinet portfolios table. 
Due to this dependency, inserting a new cabinet configuration necessitates to also insert the corresponding observations to the cabinet portfolios table.\footnote{%
Particularly, because information on the on the newly inserted cabinet's portfolios is required to generate indicators at the level of political configuration (i.e., the cabinet's cumulated seat share in the lower house and upper house, respectively, or to identify whether a president is in cohabitation with the cabinet).}

Please refer to the \texttt{PostgreSQL} documentation for further details.\footnote{See \url{https://www.postgresql.org/docs/9.3/static/dml-insert.html}}

  \subsection{Manually updating data}\label{subsec_manually_updating_data}

Altering the values of an existing row in a table is achieved with the \texttt{UPDATE}-operation, specifying the table and the column of the values that is thought to be updated.\footnote{\url{https://www.postgresql.org/docs/9.3/static/dml-update.html}}
Updating is achieved by \texttt{SET}ting a column equal to some value that matcces the type of the respective column.
A \texttt{WHERE}-clause is requiered to identify the row(s) which you attempz to update. 

A minimum working example reads as follows:
\begin{lstlisting}[language=postgreSQL]
UPDATE config_data.cabinet 
	SET cab_sdate = '2017-06-15'::DATE 
	WHERE cab_id = 6038 
	AND ctr_id = 6 
	AND cab_sdate = '2017-01-01'::DATE;
\end{lstlisting}

Here, the value of the column that reports the cabinet's start date is updated in only one observation, as the attributes \texttt{cab\_id}, and \texttt{ctr\_id} and \texttt{cab\_sdate}, respectively, uniquely identify rows in the cabinet table. (Note that using one identifier only would suffice.)

Note that it is possible to update information of more than one row. 
You could, for instance, 
\begin{lstlisting}[language=postgreSQL]
UPDATE config_data.cabinet 
	SET cab_hog_n = 'John Doe'::NAME 
	WHERE cab_hog_n = 'Licht'
	AND ctr_id = 6;
\end{lstlisting}
which would apply to all German cabinet configurations in which some guy with last name `Licht' was recorded as head of government (i.e., prime minister).

Note further that updating is proceeded row-by-row. Executing
\begin{lstlisting}[language=postgreSQL]
UPDATE config_data.cabinet 
	SET cab_id = cab_id+1
	WHERE ctr_id = 6;
\end{lstlisting}
would thus prompt an error, because increasing the first rows identifier by one would conflict with the \texttt{PRIMARY KEY}-constraint on the second rows \texttt{cab\_id}.\footnote{Becasue the second row might have \texttt{cab\_id = 6002}, increasing the first cabinet's identifier to \texttt{6002} violate the \texttt{UNIQUE}-constraint that is implicit to \texttt{PRIMARY KEY}.}


  \subsection{Manually deleting data}\label{subsec_manually_deleting_data}

Removing rows from a table is achieved with the \texttt{DELETE}-operation, specifying the table and the row to be delete.\footnote{\url{https://www.postgresql.org/docs/9.3/static/dml-delete.html}}
Deleting is achieved by identifying the row in a \texttt{WHERE}-clause. 

See the minimum working example:
\begin{lstlisting}[language=postgreSQL]
DELETE FROM config_data.cabinet 
	WHERE cab_id = 6038 
	AND ctr_id = 6 
	AND cab_sdate = '2017-06-15'::DATE;
\end{lstlisting}

This will delete the complete row from the cabinet table that is identified by \texttt{cab\_id = 6038}, that is, the (unique) German cabinet configuration that was recorded as starting on Jule 15, 2017. (Note that using one identifier only would suffice.)

Note again that it is possible to delet more than one row. 
You could, for instance, execute
\begin{lstlisting}[language=postgreSQL]
DELETE FROM config_data.cabinet 
	WHERE ctr_id = 6 AND cab_hog_n = 'John Doe';
\end{lstlisting}
in order to delete all German cabinet configurations in which some guy with last name `John Doe' was recorded as head of government (i.e., prime minister).

Note further that deleting is irreversible unless a back-up copy of the data exists (or is generated on delete).

  \subsection{Insert and update using the `upsert' function}\label{subsec_upsert_data_function}





Because the upsert-function is at the heart of the updating process, it follows a detailed description of its definition:

\begin{itemize}
\item[-]{Lines 3-5: 
the function is defined in the \texttt{public} schema of the \texttt{polconfdb} database.
It has four input arguments:
  \begin{itemize}
  \item[]{\texttt{target\_schema}: schema name of the table that is upserted (target)}
  \item[]{\texttt{target\_table}: name of the table that is upserted (target)}
  \item[]{\texttt{source\_schema}: schema name of the table that is the source of the upserted operation}
  \item[]{\texttt{source\_table}: name of the table that is the source of the upserted operation}
  \end{itemize}
All input arguments have require type \texttt{TEXT}.
}
\item[-]{Line 6: return type is \texttt{VOID}, i.e., nothing is returned}
\item[-]{Line 8: \texttt{DECLARE} variable that will be used in \texttt{EXECUTE} block (lines 48-69)}
\item[-]{Lines 9-12: variable \texttt{pkey\_column} stores the name of the column that contains the primary key of the target table}
\item[-]{Lines 14-17: variable \texttt{pkey\_constraint} stores the name of the primary key constraints of the target table}
\item[-]{Lines 19-30: array \texttt{shared\_columns} stores a comma-seperated list of the columns the target and source tables have in common; will be used in \texttt{INSERT}-statement (lines 58 and 59)}
\item[-]{Lines 32-46: array \texttt{update\_columns} stores a comma-seperated list of target columns that are set equal to source columns in \texttt{SET}-statementto of update operation (line 50)}
\item[-]{Line 48: \texttt{EXECUTE} block starts here}
\item[-]{Lines 49-56: exectue \texttt{UPDATE} target table, setting target column values equal to source column values for all intersecting identifiers (\texttt{WHERE}-clause, lines 52-54)}
\item[-]{Lines 58-64: execute \texttt{INSERT INTO} target table, inserting data into from source table for all rows that are not in target table (set difference of identifiers)}
\item[-]{Line 66: cluster data, i.e., order by priamary key values}
\end{itemize}

\lstinputlisting[%caption={Code to compute the minimum-fragmentation Effective Number of Parties in Parliament.},%
language=postgreSQL,commentstyle=\color{white}]%
{../Updates/fun_update_base_table.sql}