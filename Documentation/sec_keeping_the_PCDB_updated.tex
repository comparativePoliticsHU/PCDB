\section{Keeping the PCDB updated}\label{sec_keeping_the_PCDB_updated}
I may also provide a guide how to insert, update, and delete data from the tables contained in the PCDB.
I have not yet developed any tool to insert data, e.g., from exel tables. Inserting a mass of data is thus far proceded manually, using SQL, and often painstacking.

%\lstinputlisting[%caption={Code to determine if the president constitutes an open veto point in a configuration.},
%language=SQL]%
%{../SQL-codes/view_configuration_vto_prs.sql}

The following paragraphs will use table Cabonet as an example to introdcue some minimal working examples (MWE) thought illustrate how data is inserted inot and deleted from the tables in the confgi\-data scheme, and how recorded date can be updated

THE MWEs can easily be transfered to the other base tables in the PCDB. 
However, it is imperative to stress that no data should be cahnged without having a clear idea of 
\begin{itemize}
\item[(a)]what is the primary key of a given table or the columns that uniquely identify rows;
\item[(b)]which referntial dependencies are implied by the structure of the PCDB; and accordingly,
\item[(c)]how incomplete insertation or updating, or thoughless deletion affects the integrity and constistency of the PCDB.
\end{itemize}

With respect to the MWE, 
(a) \texttt{\footnotesize cab\_id} is primary key of table Cabinet while additionally \texttt{\footnotesize cab\_sdate} in combination with \texttt{\footnotesize ctr\_id} uniquely identify observations, i.e., rows. 

With reespect to (b), \texttt{\footnotesize cab\_id} is referenced as foreign key in table Cabinet Portfolios and, in combination with \texttt{\footnotesize pty\_id}, uniquely identifies cabinet portfolios; table Cabinet being a base-table, \texttt{\footnotesize  cab\_id}s are sequenced in the Configuration view and thus are essential to compute configuration-specific indicators, such as veto constellations, and cabinet-parties seat share in the lower and upper houses; and \texttt{\footnotesize cab\_id}s are selected by several triggers to identify previous or subsequent cabinets for any given cabinet (subsections \ref{} and \ref{}).

Lastly, in view of (c), though it is possible to insert a new observation to table Cabinet without providing, for instance, its start date, this would cause non-trivial problems in compiling view Configurations and selecting it as next cabinet for the preceeding cabinet configuration, to name but few. Users are thus strongly inclined to pay attention to the key and uniquness characteristics of a given table when inserting, updating or deleting data from it-


\subsection{Insert}

Adding new row (i.e., an observation) to a table is proceeded with the \texttt{\footnotesize INSERT INTO}- command, specifying first the table, second the columns, and third the values to insert. Though insertation does not requiere to specify the destination-columns   when using the original order of columns of a table as default, specification is best-practice, as it guarantees for correctness of the procedure.
A MWE reads as follows:

\begin{lstlisting}[language=postgreSQL]
INSERT INTO config_data.cabinet
	(cab_id, ctr_id, cab_sdate, cab_hog_n, cab_care)
	VALUES (1036, 1, '2014-01-01', 'Abbott', 'FALSE');
\end{lstlisting}

Note that the values thought to insert need to match the specified types of the destination-columns. Typing instead 

\begin{lstlisting}[language=postgreSQL]
INSERT INTO config_data.cabinet
	(cab_id, ctr_id, cab_sdate, cab_hog_n, cab_care)
	VALUES (1036::NUMERIC(5,0), 1::SMALLINT, '2014-01-01'::DATE, 'Abbott'::NAME, 'FALSE'::BOOLEAN);
\end{lstlisting}

would thus avoid any surprises.

If one attempts to insert a value that does not match the type of the respective column, pgAdmin3 notes the error and states 
To recall the type of a given column, refer to the Codebook or browse the properties of the given table in  \texttt{\footnotesize pgAdmin3} (right click on table in menu bar, querying "")

It is generally recomended to refer to either the Codebook or Section X of the Manual, before inserting data into tables, as there are set constraints (e.g.,  \texttt{\footnotesize NOT NULL},  \texttt{\footnotesize PRIMARY KEY}, or  \texttt{\footnotesize UNIQUE}) on some of the columns that make insertation of a value obligatory when adding a new row to the table.

In addition, it is best-practice to assign ascending integer counters  to subsequent instituion configurations withn countries, thought the trigger structure that assigns identifiers of previous and next configurations to a current configuration does not require this order (see subsections \ref{trg_prv_ids} and \ref{trg_nxt_ids}).

Finally, remember that the primary key of the cabinet table, \texttt{\footnotesize cab\_id}, contributes to the unambigous identification of observations in the Cabinet Portfolio table. Following the tree of dependencies, inserting a new cabinet should be followed by specifying the correspoonding cabinet portfolio.
Also, information on the on the newly inserted cabinet's portfolio is requiered to obtained meaningful information on the political configuration (i.e., the lower house, upper house, and/or presidency cabient parties face) in which it is embedded.

\subsection{Update}

Altering the values of an existing row in a table is proceeded with the \texttt{\footnotesize UPDATE}-command, specifying the table and the column of the values that is thought to be updated. 
Updating is achieved by \texttt{\footnotesize SET}ting a column equal to some value that mathces the type of the respective column.
A \texttt{\footnotesize WHERE}-clause is requiered to identify the row(s) which are ment to be updated.
A MWE reads as follows:

\begin{lstlisting}[language=postgreSQL]
UPDATE config_data.cabinet 
	SET cab_sdate = '2014-03-15'::DATE 
	WHERE cab_id = 1036 
	AND ctr_id = 1 
	AND cab_sdate = '2014-01-01'::DATE;
\end{lstlisting}

Here, the value of the column that reports the cabinet's start date is updated in only one observation, as the attributes \texttt{\footnotesize cab\_id}, and \texttt{\footnotesize cab\_id} and \texttt{\footnotesize cab\_id}, respectively, uniquely identify rows in the Cabinet table.

It is possible to update information of more than one row.


