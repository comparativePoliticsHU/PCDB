\documentclass{article}
\usepackage[utf8]{inputenc}
\usepackage[ngerman]{babel}
\usepackage{tikz}
\usetikzlibrary{positioning,shadings}
\usetikzlibrary{arrows}
\begin{document}
\begin{tikzpicture}[auto, node distance = 0.4cm, thick,
  every node/.style = {rectangle, font = \sffamily, white,
    top color = green!90!black, bottom color = green!60!black,
    text width = 2.4cm, align = center, minimum height = 1cm}]
  \node (GF)                                 {\textbf{GF}\\Gesch�ftsf�hrer};
  \coordinate [below = 0.9cm of GF] (Mitte);
  \coordinate [below = 1.6cm of GF] (Unten);
  \node (Verwaltung) [left  = 2mm of Mitte]  {Verwaltung /\\ B�ro};
  \node (EDV)        [right = 2mm of Mitte]  {EDV \& QS};
  \node (Bau)        [below = of Verwaltung] {Bau};
  \node (Logistik)   [left  = of Bau]        {Logistik};
  \node (Pflege)     [below = of EDV]        {Pflege};
  \node (Ausbildung) [right = of Pflege]     {Ausbildung};
  \draw [green!60!black,thick]
    (GF)    -- (Mitte) -- (Verwaltung)
    (EDV)   -- (Mitte) -- (Unten) -| (Logistik)
    (Unten) -| (Bau)
    (Unten) -| (Pflege)
    (Unten) -| (Ausbildung);
\end{tikzpicture}
\end{document}