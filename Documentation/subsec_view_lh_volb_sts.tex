\subsection{Type B Volatility in Lower House Seat Shares}\label{view_lh_volb_sts}
View \texttt{\footnotesize view\_lhelc\_volb\_sts} is based on tables Lower House, Lower House Seat Results and Party, and provides data at the level of lower houses.

Type B volatility quantifies the change that occurs in the distribution of seat shares within parties in subseqent lower houses, comparing the results in the current to that of the previous one. 
Accordingly, type B volatility considers only so-called stable parties and measures the  volatility in the distribution of seats arising from gaines and losses of these stable parties.	

The formular to compute \texttt{\footnotesize lh\_volb\_sts} is
\begin{equation}
Seat\,B\,Volatility\,(k) = \frac{ | \sum\limits_{j=1}^{Stable} s_{j,(k-1)} - s_{j,k}| }{2}
\end{equation}, where $s$ are seat or vote shares that party $j$ gained in the current lower house $k$ or in the previous lower house $k-1$.


View \texttt{\footnotesize view\_lh\_volb\_sts} is programmed as follows:

\lstinputlisting[%caption={Code to compute type B volatitlity ins lower house seats.},%
language=SQL]%
{../SQL-codes/view_lh_volb_sts.sql}

Stable parties are identified computationable by calculating the cross-product between rows in the subqueries 
\texttt{\footnotesize CUR\_LH\_STS\_SHR} and \texttt{\footnotesize PREV\_LH\_STS\_SHR}, and reporting only those for which a party identifier is enlisted in both the previous and the current election (cf. corresponding \texttt{\footnotesize WHERE}-clause). 

{\bf Note}: The concept of stable party makes no sense for first recorded lower houses, and hence B volaities are not computed. 
The measure is highly sensitive to missing data, as no aggregate value is computed for lower house  elections in which at least one party except the group `Others withour seat' has NULL records for total seat results (cf. consistency check \texttt{\footnotesize cc\_missing\_lh\_pty\_sts\_records} [\ref{cc_missing_lh_pty_sts_records}]). A lack of reliable lower-level data thus causes severe lack of aggregate data. 

Generally, consistency check \texttt{\footnotesize cc\_lh\_volb\_sts} [\ref{cc_lh_volb_sts}]) provides for a comparison of the computed and the recorded figures, though the recorded have been computed manually as well.

