\subsection{Cabinet}\label{subsec_tab_cabinet}
Table \texttt{\footnotesize cabinet} contains information on cabinets.
Rows are the different cabinet configurations, identified by variable \texttt{\footnotesize cab\_id}. 
A new cabinet is enlisted if one of the following events took place:
\begin{itemize}\label{cabinet_change_criteria}\itemsep-4pt \parsep0pt
\item[a)] Coalition composition changes at the party-level.
\item[b)] Head of government changes.
\item[c)] Government formation after general legislative elections (not in presidential systems).
\end{itemize}

\subparagraph{Cabinet start date} Variable \texttt{\footnotesize cab\_sdate} refers to the date on which the cabinet, as proposed by the Head of Government, recieves a vote of confidence in the legislature. The variable \texttt{\footnotesize cab\_src} regularly contains links to the websites or online repositories which are used as references. If available, data was compiled directly from information reported on government websites or other official sources.

\subparagraph{Total number of cabinet portfolios} In the present version of the database (!) the number of cabinet portolios is an integer counter equal to the number of parties in cabinet. 
Because it is an aggregate of data contained in the Cabinet Portfolios table (\ref{subsec_tab_cabinet_portfolios}), the total number of cabinet portfolios is cumputed in \texttt{\footnotesize view\_pty\_cab\_sts} (\ref{view_pty_cab_sts}).

%\subparagraph{Sources} Information is obtained from \citet*{Woldendrop_et_al2000} and the Political Data Yearbook \citeyearpar{EJPR_PDY}, and was complemented by individual-case research.

Table \texttt{\footnotesize cabinet} is defined as follows:

\lstinputlisting[%caption={Code to compute the minimum-fragmentation Effective Number of Parties in Parliament.},%
language=postgreSQL]%
{../SQL-codes/tab_cabinet.sql}

