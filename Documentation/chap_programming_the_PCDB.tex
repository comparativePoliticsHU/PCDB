\chapter{Programming the PCDB}\label{chap_programming_the_PCDB}

This chapter provides the code and corresponding explanatory descriptions of the data structre in the PCDB. 

Four types of objects will be discussed in succession:
\begin{itemize}
\item[(1)]{{\bf Tables}: The permanent data repositories that store information at different levels (e.g., parties, institutions, countries, etc.) and serve as priamry source for all computed indices and aggregate figures.}
\item[(1)]{{\bf Views}: Virtual tables based on the result-sets of defined SQL-queries.
Views serve two purposes in the PCDB:
\begin{itemize}
\item[i.]Compute aggregates and indices from the primary data contained in tables,
\item[ii.]and create consistency checks that allow o control for the consistency of the data and to trace coding failures.
\end{itemize}}
\item[(3)]{{\bf Materialized views}: Tables created from views that may be updated from the original base tables from time to time.
\item[(3)]{{\bf Triggers}: functions implemented on tables or materialized views to insert, update, or delete data as consequence of specific events. Triggers are mainly implemented to enable the automatic up-dating of the PCDB.}
\end{itemize}
