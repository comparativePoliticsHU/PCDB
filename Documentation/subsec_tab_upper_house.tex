\subsection{Upper House}\label{subsec_tab_upper_house}
Table \texttt{\footnotesize upper\_house} provides basic information on upper houses, including start date of legislature and the total number of seats. 
Rows are compositions of upper houses, identified by \texttt{\footnotesize uh\_id} as well as unique combinations of \texttt{\footnotesize ctr\_id} and \texttt{\footnotesize uh\_sdate}.
 
A new upper house composition is included when
\begin{itemize}\itemsep-4pt \parsep0pt
\item[a)]the composition changes through legislative elections, or
\item[b)]mergers or splits in factions occur during the legislature.
\end{itemize} 
Obviously, information is only provided for countries with bicameral systems.

\subparagraph{Upper house start date}
PCDB codes the date of the first meeting in the first legislative session of a new upper house as its start date. If no information on these events was available, the default is equal to the corresponding election date. 
%The table contains the following variables:

Table \texttt{\footnotesize upper\_house} is defined as follows:

\lstinputlisting[%caption={Code to compute the minimum-fragmentation Effective Number of Parties in Parliament.},%
language=postgreSQL]%
{../SQL-codes/tab_upper_house.sql}

