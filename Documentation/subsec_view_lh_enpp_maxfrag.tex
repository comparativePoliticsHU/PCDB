\subsection{Effective Number of Parties in Parliament, Maximum Fragmentation}\label{view_lh_enpp_maxfrag}
View \texttt{\footnotesize view\_lh\_enpp\_maxfrag} is based on tables Lower House and Lower House Seat Results, view ENPP Adjustment Parameter (m) (\ref{view_lh_m}), and provides data at the level of lower houses.

The effective number of parties in parliament (ENPP) is a measure of party system fractionalization that takes into acount the relative size of parties present in a country's lower house. 

Variable \texttt{\footnotesize lh\_enpp\_maxfrag} adjusts for the tendency of underestmating fractionalization of lower houses that implicite in Laakso and Taagepera's original formular (Equ \ref{ENPP_equ_minfrag}). 

It employs what \citet[pp.\,600-602]{Gallagher&Mitchell2005} refer to as `Taagepera's least component approach': The seat share of the groups `Others with seats' (otherw) and `indpendents' (INDs) are split into $m$ fractions each, resulting in $m$ seat shares of size $s_{m}$. 

The fromula to compute \texttt{\footnotesize lh\_enpp\_maxfrag} is
\begin{equation}\label{ENPP_equ_maxfrag}
ENPP_{maxfrag}(k) = 1/\sum\limits_{j=1}^{J}m\left(\frac{s_{j,k}}{m}\right)^{2}
\end{equation} 
, where $m$ is computed by dividing the number of seats of otherw or that of INDs by the number of seats of the smallest `real'\footnote{`Real' in the sense that the respective party is identified by a counter different from \#\#997 or \#\#998 (see table Party).} party in the respective lower house, and upround to the next bigger integer value, to guarantee that the seat share of otherw and/or of INDs are smaller than that of the smallest `real' party. In fact, this procedure implies assuming maximum fragmentation.

View \texttt{\footnotesize view\_lh\_enpp\_maxfrag} is programmed as follows:

\lstinputlisting[%caption={Code to compute the maximum-fragmentation Effective Number of Parties in Parliament.},%
language=SQL]%
{../SQL-codes/view_lh_enpp_maxfrag.sql}

{\bf Note}: Computation of the ENPP is proceeded with the computed, not the recorded total number of Lower House seats (note that the computed sum of seats might deviate from the recorded figure in table Lower House; cf. \texttt{\footnotesize cc\_lhelc\_sts\_ttl}).

