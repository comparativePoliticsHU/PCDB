\subsection{ENPP Adjustment Parameter (m)}\label{view_lh_m}
View \texttt{\footnotesize view\_lh\_m} is based on table Lower House Seat Results and provides data at the level of lower houses.

It computes the parameter $m$, which is used to account for the tendency to underestimate the effective number of parties in parliament (ENPP) in Laakso and Taagrepera's original formular (Equ \ref{ENPP_equ_minfrag}) 
in cases where the number of lower house seats hold by others and indpendents exceeds the number of sears hold by the smallest `real' \footnote{`Real' in the sense that the respective party is identified by a counter different from \#\#997 or \#\#998 (see table Party).} party.

Specifically, \citet[pp.\,600-602]{Gallagher&Mitchell2005} suggest to devide the total share of seats of the groups others and independents, respectively, in $m$ parts of equal size $s_{m}$, so that the $s_{m}$s are smaller than the share of the smallest party (referred to as `Taagepera's least component approach').

Accordingly, $m$ is computed by dividing the seats of (a) the groups `Others with seat' (Otherw) and/or `Independents' (IND) hold in the lower house by the number of seats (b) the smallest 
`real' party holds in the respective lower house.
When (a) > (b), then $m > 1$.
To guarantee that the $m$ seat shares $s_{m}$ of Otherw and/or IND is/are smaller than that of the smallest party, $m$ is upround to the next bigger integer value.
Lower House elections in which m is bigger than one are enlisted in \texttt{\footnotesize view\_lhelc\_w\_underestimated\_ENPP} (\ref{view_lhelc_w_underestimated_ENPP}).


View \texttt{\footnotesize view\_lh\_m} is programmed as follows:

\lstinputlisting[%caption={Code to compute the ENPP adjustement parameter m.},%
language=SQL]%
{../SQL-codes/view_lh_m.sql}

{\bf Note}: The \texttt{\footnotesize WHERE}-condition ensures that only lower house elections are selected in which the amount of seat of Others with seats or Independents exceedes the amount of seats the party with least seats gained, as `Taagepera's least component approach' to prevent from underestimation of ENPPs only needs to be applied to these cases.

