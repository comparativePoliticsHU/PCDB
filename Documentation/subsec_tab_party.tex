\subsection{Party}\label{subsec_tab_party}
Table \texttt{\footnotesize party} provides basic information on parties, permitting to link them to other party-level databases or tables in the PCDB. 
Rows are parties within countries, identified by unique combinations of  \texttt{\footnotesize ctr\_id} and \texttt{\footnotesize pty\_id}.

\subparagraph{Party identifier}
The PCDB uses simple running counters to identify parties in a country's political system and history (variable \texttt{\footnotesize pty\_id}).
That is, in contrast to the coding schemes applied in the Manifesto Project \citep{ManifestoData2013} or the ParlGov data \citep{ParlGov2012}, identifiers do not encode allignment with party-families or ideological leaning on a left-right scale.

Special suffix are assigned to
independent candidates (\#\#997), 
other parties with seats (\#\#998), and 
other parties without seats in the legislature (\#\#999).
  
Table \texttt{\footnotesize party} is defined as follows: 

\lstinputlisting[%caption={Code to compute the minimum-fragmentation Effective Number of Parties in Parliament.},%
language=postgreSQL]%
{../SQL-codes/tab_party.sql}



