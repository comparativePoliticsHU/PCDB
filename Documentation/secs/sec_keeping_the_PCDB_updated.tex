\section{Keeping the PCDB updated}\label{sec_keeping_the_PCDB_updated}

Data in the PCDB is manipulated using \texttt{PostgreSQL}'s data manipulation language (DML) operations \texttt{INSERT}, \texttt{UPDATE}, and \texttt{DELETE}. \footnote{\url{https://www.postgresql.org/docs/9.3/static/dml.html}}

The following paragraphs will use the cabinet table (see subsection \ref{subsec_tab_cabinet}) in the \texttt{config\_data} schema of the \texttt{polconfdb} database as an example to introdcue some minimal working examples.

These examples can easily be applied to the other tables in the PCDB. 

\paragraph{Some words of caution} 
Please do not manipulate (i.e., insert, update, or delete) data without having a clear idea of 
\begin{itemize}
\item[a)]what is the primary key of a given table or the columns that uniquely identify rows;
\item[b)]which referential dependencies are implied by the structure of the PCDB; and accordingly,
\item[c)]how incomplete inserts or updates, or thoughtless delets affects the integrity and constistency of the PCDB.
\end{itemize}
Read about primary keys and the implementation of referential dependcies using foreing keys in the \texttt{PostgreSQL} documentation.\footnote{\url{https://www.postgresql.org/docs/9.3/static/ddl-constraints.html}}

With respect to the minium workin example, 
(a) The cabinet identifiers column (\texttt{cab\_id}) is primary key of cabinet table, and cabinet start date (\texttt{cab\_sdate}) in combination with the country identifier (\texttt{ctr\_id}) uniquely identify observations (i.e., rows). 

With reespect to (b), \texttt{cab\_id} is referenced as foreign key in the cabinet portfolios table (see subsection \ref{subsec_tab_cabinet_portfolios}), and, in combination with the party identifier \texttt{pty\_id}, uniquely identifies cabinet portfolios
Moreover, as cabinet compositions (i.e., rows in the cabinet tables) sequenced alongside lower house, upper house, and presidency configurations in the configuration events view, cabinet compositions are essential to compute configuration-specific indicators, such as cabinet parties cumulated seat share in the lower house; to identify open veto points; etc.
%Lastely, and \texttt{\footnotesize cab\_id}s are selected by several triggers to identify previous or subsequent cabinets for any given cabinet (subsections \ref{} and \ref{}).

Finally, in view of (c), though it is possible to insert a new observation to table Cabinet without providing, for instance, its start date, this would cause non-trivial problems, for instance, when compiling the configurations events view.

Users are thus strongly inclined to pay attention to the key and uniquness constraints of a given table when inserting, updating or deleting data from it. Information on constraints is provided in the respective subsections of the Table section (\ref{sec_table}) and the PCDB Codebook (see documentation Appendix).

\paragraph{Some words on data consistency}

Note that the trigger structure and functions defined on the \texttt{config\_data} schema ensures that manipulation executed on the cabinet, lower house, upper house, presidential election, and veto points tables propgate through to the configuration events and configuration country-years tables.
The interrelation between the configuration tables and the structure is explained in detail in sections \ref{sec_views_in_config_data_scheme}, \ref{sec_mviews_in_config_data_scheme} and \ref{sec_triggers}.

In other cases, such as the interrelation between the cabinet portfolios on the cabinet table, dependencies exist, but consistency is not enforced using a trigger structure. If you insert a new cabinet configuration, you have to manually add the corresponding cabinet portfolio (rows of parties in cabinet and the parliamentary opposition). No error will be raised if you fail to do so.
Likewise, if you record a new lower house election (upper house election), you have to make sure that the corresponding vote results are listed at the party level in the lower house vote results table, and that you record the lower house (upper house) configuration that corresponds to the election.
And if you record a new lower house (upper house) composition, you have to make sure that the corresponding seat results are listed at the party level in the lower house seat results (upper house seat results) table. 

  \subsection{Manually inserting data}\label{subsec_manually_inserting_data}

Adding a new row (i.e., an observation) to a table is proceeded with the \texttt{INSERT INTO}-command, by simply specifying the table (and schema), then the target columns, and third the values to insert.
Though insertation does not requiere to specify the target columns, as the original order of columns of a table is used as default, specifying target columns corresponding to insert values is best-practice, as it ensures a correct insert operation.

Here a minimum workin example:

\begin{lstlisting}[language=postgreSQL]
INSERT INTO config_data.cabinet
	(cab_id, ctr_id, cab_sdate, cab_hog_n, cab_care)
	VALUES (6038, 6, '2017-01-01', 'Licht', 'FALSE');
\end{lstlisting}

Note that the values you attempt to insert need to match the specified types of the target columns. 
If you attempt to insert a value that does not match the type of the respective column, an error message will be raised.\footnote{%
To recall the type of a given column, refer to the Codebook or browse the properties of the given table in \texttt{pgAdmin3} (left click on table in menu bar, and view `SQL pane').}
You can avoid such error messages, if you type instead 

\begin{lstlisting}[language=postgreSQL]
INSERT INTO config_data.cabinet
	(cab_id, ctr_id, cab_sdate, cab_hog_n, cab_care)
	VALUES (6038::NUMERIC(5,0), 6::SMALLINT, '2017-01-01'::DATE, 
    'Licht'::NAME, 'FALSE'::BOOLEAN);
\end{lstlisting}

Always refer to either the Codebook or browse the properties of the given table in \texttt{pgAdmin3} before you attempt to insert data into a table, as there exist constraints (e.g., \texttt{NOT NULL}, \texttt{PRIMARY KEY}, or \texttt{UNIQUE}) on some of the columns, which require inserting a value to these specific columns when adding a new row to the table.

Also, it is best-practice to assign ascending integer counters  to subsequent instituion configurations withn countries.
Finally, remember that the primary key of the cabinet table, \texttt{cab\_id}, contributes to the unique identification of observations in the cabinet portfolios table. 
Due to this dependency, inserting a new cabinet configuration necessitates to also insert the corresponding observations to the cabinet portfolios table.\footnote{%
Particularly, because information on the on the newly inserted cabinet's portfolios is required to generate indicators at the level of political configuration (i.e., the cabinet's cumulated seat share in the lower house and upper house, respectively, or to identify whether a president is in cohabitation with the cabinet).}

Please refer to the \texttt{PostgreSQL} documentation for further details.\footnote{See \url{https://www.postgresql.org/docs/9.3/static/dml-insert.html}}

  \subsection{Manually updating data}\label{subsec_manually_updating_data}

Altering the values of an existing row in a table is achieved with the \texttt{UPDATE}-operation, specifying the table and the column of the values that is thought to be updated.\footnote{\url{https://www.postgresql.org/docs/9.3/static/dml-update.html}}
Updating is achieved by \texttt{SET}ting a column equal to some value that matcces the type of the respective column.
A \texttt{WHERE}-clause is requiered to identify the row(s) which you attempz to update. 

A minimum working example reads as follows:
\begin{lstlisting}[language=postgreSQL]
UPDATE config_data.cabinet 
	SET cab_sdate = '2017-06-15'::DATE 
	WHERE cab_id = 6038 
	AND ctr_id = 6 
	AND cab_sdate = '2017-01-01'::DATE;
\end{lstlisting}

Here, the value of the column that reports the cabinet's start date is updated in only one observation, as the attributes \texttt{cab\_id}, and \texttt{ctr\_id} and \texttt{ cab\_sdate}, respectively, uniquely identify rows in the cabinet table. (Note that using one identifier only would suffice.)

Note that it is possible to update information of more than one row. 
You could, for instance, 
\begin{lstlisting}[language=postgreSQL]
UPDATE config_data.cabinet 
	SET cab_hog_n = 'John Doe'::NAME 
	WHERE cab_hog_n = 'Licht'
	AND ctr_id = 6;
\end{lstlisting}
which would apply to all German cabinet configurations in which some guy with last name `Licht' was recorded as head of government (i.e., prime minister).

Note further that updating is proceeded row-by-row. Executing
\begin{lstlisting}[language=postgreSQL]
UPDATE config_data.cabinet 
	SET cab_id = cab_id+1
	WHERE ctr_id = 6;
\end{lstlisting}
would thus prompt an error, because increasing the first rows identifier by one would conflict with the \texttt{PRIMARY KEY}-constraint on the second rows \texttt{cab\_id}.\footnote{Becasue the second row might have \texttt{cab\_id = 6002}, increasing the first cabinet's identifier to \texttt{6002} violate the \texttt{UNIQUE}-constraint that is implicit to \texttt{PRIMARY KEY}.}


  \subsection{Manually deleting data}\label{subsec_manually_deleting_data}

Removing rows from a table is achieved with the \texttt{DELETE}-operation, specifying the table and the row to be delete.\footnote{\url{https://www.postgresql.org/docs/9.3/static/dml-delete.html}}
Deleting is achieved by identifying the row in a \texttt{WHERE}-clause. 

See the minimum working example:
\begin{lstlisting}[language=postgreSQL]
DELETE FROM config_data.cabinet 
	WHERE cab_id = 6038 
	AND ctr_id = 6 
	AND cab_sdate = '2017-06-15'::DATE;
\end{lstlisting}

This will delete the complete row from the cabinet table that is identified by \texttt{cab\_id = 6038}, that is, the (unique) German cabinet configuration that was recorded as starting on Jule 15, 2017. (Note that using one identifier only would suffice.)

Note again that it is possible to delet more than one row. 
You could, for instance, execute
\begin{lstlisting}[language=postgreSQL]
DELETE FROM config_data.cabinet 
	WHERE ctr_id = 6 AND cab_hog_n = 'John Doe';
\end{lstlisting}
in order to delete all German cabinet configurations in which some guy with last name `John Doe' was recorded as head of government (i.e., prime minister).

Note further that deleting is irreversible unless a back-up copy of the data exists (or is generated on delete).

  \subsection{Insert and update using the `upsert' function}\label{subsec_upsert_data_function}

Suppose you have created a CSV table with, say, cabinet configuration that contains both new cabinet configurations and, in addition, changes to already existing cabinets. That is, the listed cabinet configurations in your table may match some recorded cabinet configurations in the PCDB cabinet table. 

Due to the \texttt{UNIQUE}-constraint on \texttt{ctr\_id} and \texttt{cab\_sdate} in the cabinet table, attempting to insert cabinet configurations that are identified by an already recorded \texttt{ctr\_id}-\texttt{cab\_sdate} combination would prompt an error. 
And its likely that, while you want to add not-yet recorded configurations to the cabinet table, you simply want to update the already existing configuration in the PCDB where the information on a given configuration on in your table differs from that in the current record.

This scenario is where the \texttt{upsert}-function comes in to play (`upsert' stands for update-or-insert).
Plainly speaking, the upsert function performs exaclty the steps outlined in the above paragraph: 
First, it takes your table as source of the upsert operation, checking which columns actually correspond to the columns of the target table (i.e., the table you want to populate with your new records). 
Second, the function checks if a record in the source table matches a record (i.e., row) in the target table.
The result of this second check are two distinct result sets (your source table is split into two categories, so to speak): One containing all observations that are not yet recorded in the target table. This first result set is the base of a grand insert operation on the target table.
The other result set comprises all observations in the source table that are already recorded in the target table, and hence is the base of a grand update opertion on the target table. 

Put simply, the function looks up which column(s) contain the primary key of the target variable, and then checks if a given observation's primary-key column value in the source table exists in the target table. 
For example, \texttt{cab\_id} is the primary-key column of the cabinet table. Say your target table contains a cabinet configration with the \texttt{cab\_id} value 1040.
If ``Is 1040 is in the list of all values of the \texttt{cab\_id} column in target table?'' evaluates to true, this row in the source table will be in the second result set. Otherwise it will be in the first, insert-operations result set.





\paragraph{Function definition}

Because the \texttt{upsert}-function is at the heart of the updating process, it follows a detailed description of its definition, both in pseudo-code and the PostreSQL procedural language \texttt{plpgsql}.\footnote{See \url{https://www.postgresql.org/docs/8.4/static/plpgsql.html}}.

You may want to skip this paragraph if you are immediately interested in a minimal working example (beginning on page \pageref{par_upsert_minimal_working_example}). 
You may need to turn to the functional defintion, however, Whenever the \texttt{upsert}-function is not yielding the results you were intending it to give.

\begin{itemize}
\item[-]{Lines 3-5: 
the function is defined in the \texttt{public} schema of the \texttt{polconfdb} database.
It has four input arguments:
  \begin{itemize}
  \item[]{\texttt{target\_schema}: schema name of the table that is upserted (target)}
  \item[]{\texttt{target\_table}: name of the table that is upserted (target)}
  \item[]{\texttt{source\_schema}: schema name of the table that is the source of the upserted operation}
  \item[]{\texttt{source\_table}: name of the table that is the source of the upserted operation}
  \end{itemize}
All input arguments have require type \texttt{TEXT}.
}
\item[-]{Line 6: return type is \texttt{VOID}, i.e., nothing is returned}
\item[-]{Line 8: \texttt{DECLARE} variable that will be used in \texttt{EXECUTE} block (lines 48-69)}
\item[-]{Lines 9-12: variable \texttt{pkey\_column} stores the name of the column that contains the primary key of the target table}
\item[-]{Lines 14-17: variable \texttt{pkey\_constraint} stores the name of the primary key constraints of the target table}
\item[-]{Lines 19-30: array \texttt{shared\_columns} stores a comma-seperated list of the columns the target and source tables have in common; will be used in \texttt{INSERT}-statement (lines 58 and 59)}
\item[-]{Lines 32-46: array \texttt{update\_columns} stores a comma-seperated list of target columns that are set equal to source columns in \texttt{SET}-statementto of update operation (line 50)}
\item[-]{Line 48: \texttt{EXECUTE} block starts here}
\item[-]{Lines 49-56: execute \texttt{UPDATE} target table, setting target column values equal to source column values for all intersecting identifiers (\texttt{WHERE}-clause, lines 52-54)}
\item[-]{Lines 58-64: execute \texttt{INSERT INTO} target table, inserting data into from source table for all rows that are not in target table (set difference of identifiers)}
\item[-]{Line 66: cluster data, i.e., order by priamary key values}
\end{itemize}

\lstinputlisting[%caption={Code to compute the minimum-fragmentation Effective Number of Parties in Parliament.},%
language=postgreSQL,commentstyle=\color{white}]%
{../Updates/fun_update_base_table.sql}

\paragraph{A minimal working example}\label{par_upsert_minimal_working_example}

To stick with the above example of making changes to the cabinet table in the PCDB, suppose yout task is to check cabinet start dates, and to add cabinet configurations that are not yet recorded in the \texttt{config\_data} schema of the database.
Say you split the work load with your co-workers, and you start with checking and updating all Australian cabinet configurations.

\subparagraph{Exporting the to-be-updated data}
The first step in a well-organized work flow would be to export all recorded Australian cabinet configurations that require a double-check of the start date into a CSV.
The following query would give you just these configurations: 
\begin{lstlisting}[language=postgreSQL]
SELECT * FROM config_data.cabinet 
	WHERE ctr_id = 1
	AND cab_valid_sdate = FALSE;
\end{lstlisting}
Note that the column \texttt{cab\_valid\_sdate} is a boolean indicator that records whether the start date of a given cabinet configuration has already been double-checked. Hence, you only want the Australian cabinet configurations where this is not yet the case.

Exporting the result set of the query int o a CSV is easily achieved using the write-result-to-file wizard of \texttt{pgAdmin3}'s SQL-query tool. (Refer to Subsection \ref{subsec_sql_query_tool}, and figures \ref{fig_pgadmin3_sql_query_tool_toolbar} and \ref{fig_pgadmin3_sql_query_tool_export_data_wizard} in particular, if you do not know how to do export data to a file in \texttt{pgAdmin3}.)

In order to know which Australian cabinets are not yet recorded, and hence need to be added in your `upsert' source table, you need to know, which is the youngest recorded Austrian cabinet in the PCDB (i.e., the cabinet with the most recent start date). 
the The result set of the above query does not necessarily inform you about this,  however (if \texttt{cab\_valid\_sdate} is true for the last recorded cabinet configratuion, it will not be in the result set.)

You could query 
\begin{lstlisting}[language=postgreSQL]
SELECT * FROM config_data.cabinet 
	WHERE ctr_id = 1
	ORDER BY cab_sdate DESC
	LIMIT 1;
\end{lstlisting}
in order to get the respective information, or export all Austrian cabinet configurations in the first place, and only check start dates of these where  \texttt{cab\_valid\_sdate} is false instead.

\subparagraph{Changing the to-be-updated data}
With the exported CSV at hand, you can directly make your changes in the repective cells of the table; of course always documenting your changes and the information sources in the comment and source columns.
In case of already existing cabinets, you would not change the \texttt{cab\_id} but only the \texttt{cab\_sdate}.
In case of missing cabinets, you would choose a not existing \texttt{cab\_id} value (optimally increasing it by one within country with ascending start dates) and add all corresponding inforamtion in the respective cells of that new entry.

\subparagraph{Getting the to-be-updated data into the PCDB}
In order to upsert the target table with the changes you have recorded in your CSV, the data in your CSV first needs to be imported into the PCDB again.
The \texttt{update} schema of the PCDB provides for the environment in which you can securely import to-be-updated data into upsert source tables.

Note that, if it not already exists, you first have to create a table in the \texttt{update} schema that matches the column names and types of your CSV.
If you have proceeded as described thus far in this minimal working example, your CSV containing the to-be-updated data will have the same definition as the upsert target table (because the CSV was orginally exported from the target table).
Hence you can simply type 
\begin{lstlisting}[language=postgreSQL]
CREATE TABLE IF NOT EXISTS update.cabinet (LIKE config_data.cabinet INCLUDING ALL)
\end{lstlisting}
Note that the option \texttt{INCLUDING ALL} will create a new table that copies all column names, their data types, their not-null constraints, primary and foreign key.\footnote{See \url{https://www.postgresql.org/docs/9.1/static/sql-createtable.html}}
The resulting table will be empty but prompt the same requirments when inserting data as the target table (here cabinet in the \texttt{config\_data} schema). 
That is, you won't be able to insert duplicate \texttt{cab\_id}s, rows with missing start date information, etc. (see the definition of table cabinet for a compelte list of column and table cosntraints).

Once you have created an empty source table in the \texttt{update} schema as, you can use \texttt{pgAdmin3}'s easy-to-handle import wizard to import data to the now existing table.\footnote{%
If the table is not empty, e.g., storing data from previous updating rounds, its recommended to remove the superflous data before adding new to-be-updated data. 
Use the `Drop/Delete ...' function provided on right-click on the respective table in the Objecet Browser or explicit SQL to empty the source table.}
Simply right-click on the table in the Object Browser and select ``Import'' (See Figure \ref{fig_pgadmin3_import_data_to_table}). 
The `Import data from file into table' wizard will open (shown for the case of the cabinet table in Figure \ref{fig_pgadmin3_import_data_wizard}), and allow you to browse your system for the respective CSV. 
Remember to select check-box Header in the `Misc. Options' tab of the wizard and enter the delimiter of the data (in CSVs produced with German configuration, this is commonly the semicolon \texttt{;}).

\begin{figure}[ht!]
\centering
  \begin{subfigure}{.45\textwidth}
  \includegraphics[width=\textwidth,trim= 0 0 0 0, clip]{pcdb_documentation_screenshots/pgadmin3_import_data_to_table.png}
    \subcaption{How to open the data-import wizard for a table in \texttt{pgAdmin3}'s Object Browser.}
    \label{fig_pgadmin3_import_data_to_table}
  \end{subfigure}
  ~%
  \begin{subfigure}{.45\textwidth}
  \includegraphics[width=\textwidth,trim= 0 0 0 0, clip]{pcdb_documentation_screenshots/pgadmin3_import_data_wizard.png}
    \subcaption{The `File Options' tab of \texttt{pgAdmin3}'s data-import wizard.}
    \label{fig_pgadmin3_import_data_wizard}
  \end{subfigure} 
  \caption{\texttt{pgAdmin3}'s `Import data from file into table' wizard.}
\end{figure}

In case you have initially exported fewer columns from the target table, you can use the `Columns' tab in the wizard to unselect the columns of the source table that are not recorded in your CSV.
Alternatively, you can define a table writing explicit SQL. Refer to section \ref{sec_tables} for examples. 

\subparagraph{Upserting the target table based on the data in the source table}

Once you have exported the to-be-updated data from your target table into a CSV, made your changes in the CSV, and imported it to the source table in the \texttt{update} schema, you can call the upsert function by executing the following code in the SQL editor:

\begin{lstlisting}[language=postgreSQL]
SELECT upsert_base_table(
  target_schema='config_data', target_table='cabinet',
  source_schema='update', source_table='cabinet')
-- alternatively, but less explicit and hence more error prone
SELECT upsert_base_table('config_data', 'cabinet', 'update', 'cabinet')
\end{lstlisting}

In order to better understand the working of the \texttt{upsert}-function, lets use this minimal working example to reconstruct what's happening under the hood when executing above query.

First, it queries the primary-key information from the \texttt{constraint\_column\_usage} table in the \texttt{information\_schema} schema.

\begin{lstlisting}[language=postgreSQL]
-- parameter pkey_column will have the following value
SELECT column_name::VARCHAR FROM information_schema.constraint_column_usage 
  WHERE (table_schema = 'config_data' AND table_name = 'cabinet')
	AND constraint_name LIKE '%pkey%';
	-- returning cab_id
	
-- parameter pkey_constraint will have the following value
SELECT constraint_name::VARCHAR FROM information_schema.constraint_column_usage 
  WHERE (table_schema = 'config_data' AND table_name = 'cabinet')
  AND constraint_name LIKE '%pkey%';
	-- returning cabinet_pkey
\end{lstlisting}

Then get the intersecting columns, i.e., the columns that exist in both the target and the source table, and store the result set as comma-separated string of column names in the parameter \texttt{shared\_columns}:

\begin{lstlisting}[language=postgreSQL]
WITH intersecting_columns AS (
  SELECT column_name, ordinal_position FROM information_schema.columns 
  	WHERE table_schema = 'config_data' 
  	AND table_name = 'cabinet'
  	AND column_name IN 
  		(SELECT column_name 
  			FROM information_schema.columns 
  			WHERE table_schema = 'update'
  			AND table_name = 'cabinet')
  	ORDER BY ordinal_position)
SELECT ARRAY_TO_STRING(ARRAY(SELECT column_name::VARCHAR AS columns FROM intersecting_columns), ', ');	
\end{lstlisting}

In order to be able to set the values of the columns the target table shares with the source table equal to the values in the corresponding columns in the source table, a comma separated string is constructed following the logic \texttt{SET target\_column = source\_column}, and stored in the parameter \texttt{update\_columns}:

\begin{lstlisting}[language=postgreSQL]
WITH intersecting_columns AS (
  SELECT column_name, ordinal_position FROM information_schema.columns 
  	WHERE table_schema = 'config_data' 
  	AND table_name = 'cabinet'
  	AND column_name IN 
  		(SELECT column_name 
  			FROM information_schema.columns 
  			WHERE table_schema = 'update'
  			AND table_name = 'cabinet')
  	AND column_name NOT LIKE 'cab_id' -- which is the value stored in parameter pkey_column
  	ORDER BY ordinal_position)
SELECT ARRAY_TO_STRING(
  ARRAY(SELECT '' || column_name || ' = update_source.' || column_name FROM intersecting_columns),
  ', ');	
\end{lstlisting}

Note the use of the above declared parameter \texttt{pkey\_column} to exclude the primary-key column from the update operation. (Setting the \texttt{cab\_id} in the target table equal to \texttt{cab\_id} in the source table makes no sense, if corresponding observations in both tables are identified by eqality of  \texttt{cab\_id}.)
Also, note that prefixing the column name in the source table with \texttt{update\_source} is due to the fact that in the subsequent update operation the subquery from which the update will be performed has the alias \texttt{update\_source} (see line 55 of the function definition). 

Further It is important to note that the \texttt{upsert}-function will only perform an upsert of data in columns that have the same (i.e., intersecting) name in the source and target tables.
If you have, for instance, added an additional commenting column in your CSV, you may be able to import this column, too, by defining the source table such that it allows to import data from this additional-comments column. 
Calling the upsert function, however, will ignore this non-intersecting column.

When all required parameters are declared, concatenating the parameters values into long strings that can be called in \texttt{EXECUTE} statements allows to perform the due upsert and insert operations.
The resulting update statement reads as follows given the above declared parameters:
\begin{lstlisting}[language=postgreSQL]
EXECUTE 'UPDATE config_data.cabinet 
	SET cab_prv_id = update_source.cab_prv_id, 
		ctr_id = update_source.ctr_id, 
		cab_sdate = update_source.cab_sdate, 
		cab_hog_n = update_source.cab_hog_n, 
		cab_sts_ttl = update_source.cab_sts_ttl, 
		cab_care = update_source.cab_care, 
		cab_cmt = update_source.cab_cmt, 
		cab_src = update_source.cab_src, 
		cab_nxt_id = update_source.cab_nxt_id, 
		cab_valid_sdate = update_source.cab_valid_sdate
	FROM (SELECT * FROM update.cabinet 
		WHERE cab_id IN (SELECT DISTINCT cab_id FROM config_data.cabinet) 
		) AS update_source 
	WHERE cabinet.cab_id = update_source.cab_id';
\end{lstlisting}
Note that it is updated performed only for the set of observations that recorded in both the target and the source table. 

Conversely, the insert statement is
\begin{lstlisting}[language=postgreSQL]
EXECUTE 'INSERT INTO config_data.cabinet 
	(cab_id, cab_prv_id, ctr_id, cab_sdate, 
	cab_hog_n, cab_sts_ttl, cab_care, cab_cmt, 
	cab_src, cab_nxt_id, cab_valid_sdate) 
	SELECT cab_id, cab_prv_id, ctr_id, cab_sdate, 
	    cab_hog_n, cab_sts_ttl, cab_care, cab_cmt, 
	    cab_src, cab_nxt_id, cab_valid_sdate
		FROM (SELECT * FROM update.cabinet
			WHERE cab_id NOT IN (SELECT DISTINCT cab_id FROM config_data.cabinet)
		) AS insert_source';
\end{lstlisting}
Here, insert is only performed for the set of rows identified by \texttt{cab\_id} in the source table, whose  \texttt{cab\_id} value is {\em not} yet recorded in the target table. This is, in fact, the crux of an upsert operation: Insert only where no update possible, because no identifiable record exists. 


Please, as always, use the \texttt{beta\_version} schema for any test run of the function.





