\subsection{CC Upper House start date}
Consistency check \texttt{\footnotesize cc\_uh\_sdate} is based on tables Upper House and Upper House Election and provides information at the level of upper house (LH) configurations.

It compares a UH's start date (\texttt{\footnotesize lh\_sdate}) and the date of the UH election it emanates from (\texttt{\footnotesize uhelc\_date}). Variable \texttt{\footnotesize date\_dif} measures the difference between both dates in days; variable \texttt{\footnotesize prob\_corr\_ddif} is zero when the recorded date of lower house formation, i.e., its start date, and the election date are equal. 
Logically, this is not expecte to be the case, as a UH's start date should be later than the date of its election.

CC \texttt{\footnotesize cc\_uh\_sdate} is programmed as follows

\lstinputlisting[%caption={Code to create consistency check on upper house start dates.},%
language=SQL]%
{../SQL-codes/cc_uh_sdate.sql}

The explicit, and actually empirically reasonable assumption is the first meeting in the first session of a newly elected UH (coded as start date) is regularly not on the same day as the election but on a later date in the countries that are covered in the PCDB. 
Because in the coding process the date of the election a UH emanates from has been used as default UH start date, if the difference is equal to zero, this strongly indicates that no proper start date has been recorded. Case-specific research is still required for these UH configurations.

\emph{\textbf{Note}}: Information on the sources of UH start dates is provided in variable \texttt{\footnotesize uh\_src} of the Upper House table and variable \texttt{\footnotesize valid\_uh\_sdate}, in turn, is an individually coded dummy that indicates whether individual case research has been properly conducted an the source of information appears reliable.
