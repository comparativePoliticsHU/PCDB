\subsection{Insert and update using the `upsert' function}\label{subsec_upsert_data_function}





Because the upsert-function is at the heart of the updating process, it follows a detailed description of its definition:

\begin{itemize}
\item[-]{Lines 3-5: 
the function is defined in the \texttt{public} schema of the \texttt{polconfdb} database.
It has four input arguments:
  \begin{itemize}
  \item[]{\texttt{target\_schema}: schema name of the table that is upserted (target)}
  \item[]{\texttt{target\_table}: name of the table that is upserted (target)}
  \item[]{\texttt{source\_schema}: schema name of the table that is the source of the upserted operation}
  \item[]{\texttt{source\_table}: name of the table that is the source of the upserted operation}
  \end{itemize}
All input arguments have require type \texttt{TEXT}.
}
\item[-]{Line 6: return type is \texttt{VOID}, i.e., nothing is returned}
\item[-]{Line 8: \texttt{DECLARE} variable that will be used in \texttt{EXECUTE} block (lines 48-69)}
\item[-]{Lines 9-12: variable \texttt{pkey\_column} stores the name of the column that contains the primary key of the target table}
\item[-]{Lines 14-17: variable \texttt{pkey\_constraint} stores the name of the primary key constraints of the target table}
\item[-]{Lines 19-30: array \texttt{shared\_columns} stores a comma-seperated list of the columns the target and source tables have in common; will be used in \texttt{INSERT}-statement (lines 58 and 59)}
\item[-]{Lines 32-46: array \texttt{update\_columns} stores a comma-seperated list of target columns that are set equal to source columns in \texttt{SET}-statementto of update operation (line 50)}
\item[-]{Line 48: \texttt{EXECUTE} block starts here}
\item[-]{Lines 49-56: exectue \texttt{UPDATE} target table, setting target column values equal to source column values for all intersecting identifiers (\texttt{WHERE}-clause, lines 52-54)}
\item[-]{Lines 58-64: execute \texttt{INSERT INTO} target table, inserting data into from source table for all rows that are not in target table (set difference of identifiers)}
\item[-]{Line 66: cluster data, i.e., order by priamary key values}
\end{itemize}

\lstinputlisting[%caption={Code to compute the minimum-fragmentation Effective Number of Parties in Parliament.},%
language=postgreSQL,commentstyle=\color{white}]%
{../Updates/fun_update_base_table.sql}