\subsection{Manually deleting data}\label{subsec_manually_deleting_data}

Removing rows from a table is achieved with the \texttt{DELETE}-operation, specifying the table and the row to be delete.\footnote{\url{https://www.postgresql.org/docs/9.3/static/dml-delete.html}}
Deleting is achieved by identifying the row in a \texttt{WHERE}-clause. 

See the minimum working example:
\begin{lstlisting}[language=postgreSQL]
DELETE FROM config_data.cabinet 
	WHERE cab_id = 6038 
	AND ctr_id = 6 
	AND cab_sdate = '2017-06-15'::DATE;
\end{lstlisting}

This will delete the complete row from the cabinet table that is identified by \texttt{cab\_id = 6038}, that is, the (unique) German cabinet configuration that was recorded as starting on Jule 15, 2017. (Note that using one identifier only would suffice.)

Note again that it is possible to delet more than one row. 
You could, for instance, execute
\begin{lstlisting}[language=postgreSQL]
DELETE FROM config_data.cabinet 
	WHERE ctr_id = 6 AND cab_hog_n = 'John Doe';
\end{lstlisting}
in order to delete all German cabinet configurations in which some guy with last name `John Doe' was recorded as head of government (i.e., prime minister).

Note further that deleting is irreversible unless a back-up copy of the data exists (or is generated on delete).
