\subsection{Electoral Alliances}\label{subsec_tab_electoral_alliances}
Table \texttt{\footnotesize electoral\_alliances} provides information on electoral alliances, attempting to identify the parties forming an electoral alliance. 
Parties listed in the Party table (\ref{subsec_tab_party}) that are recorded as electoral alliances are listed in with their respective \texttt{\footnotesize pty\_id}s.

Variable \texttt{\footnotesize pty\_eal\_nbr} is a counter that enumerates parties that constitute an electoral alliance.\footnote{The counter is also recorded in the Party table and equals one for all `conventional' parties.}
Accordingly, there occur as many rows for each electoral alliance in the table as variable \texttt{\footnotesize pty\_eal} counts. 

Variable \texttt{\footnotesize pty\_eal\_id}, in turn, records the party identifiers of the parties that form an electoral alliance. Combinations of \texttt{\footnotesize pty\_id} (electoral alliance) and \texttt{\footnotesize pty\_eal\_nbr} (enumerator of party in electoral alliance) are therefore unique within countries.

\begin{table}[h!]
\centering\footnotesize
\caption*{Example: Composition of selected electoral alliances in Portugal.}
\begin{tabular}{c c *{1}{D{.}{.}{-1}} c c}
\tabularnewline\toprule\toprule
\multicolumn{3}{c}{Electoral Alliances} & \multicolumn{2}{c}{Party} \tabularnewline\addlinespace
\multicolumn{1}{c}{Identifier}	&	\multicolumn{1}{c}{Abbrevation}	&	\multicolumn{1}{c}{Enumerator}	&	\multicolumn{1}{c}{Identifier}	&	\multicolumn{1}{c}{Abbrevation}	\tabularnewline
\multicolumn{1}{c}{\texttt{\smallfont pty\_id}}	&	\multicolumn{1}{c}{\texttt{\smallfont pty\_abr}}	&	\multicolumn{1}{c}{\texttt{\smallfont pty\_eal\_n}}	&	\multicolumn{1}{c}{\texttt{\smallfont pty\_eal\_id}}	&	\multicolumn{1}{c}{ }	\tabularnewline
\midrule\addlinespace
8003	&	AP	&	1	&	8999	&	Other	\tabularnewline\addlinespace
8003	&	AP	&	2	&	8999	&	Other	\tabularnewline\addlinespace
8003	&	AP	&	3	&	8999	&	Other	\tabularnewline\addlinespace
8005	&	PSP.US	&	99	&	8058	&	PSP	\tabularnewline\addlinespace
8006	&	PDPC	&	1	&	8059	&	CDC	\tabularnewline\addlinespace
8006	&	PDPC	&	2	&	8999	&	Other	\tabularnewline\addlinespace
8006	&	PDPC	&	3	&	8999	&	Other	\tabularnewline\addlinespace
8006	&	PDPC	&	4	&	8999	&	Other	\tabularnewline\bottomrule\bottomrule\addlinespace
\end{tabular}
\end{table}

The example displays a selection from the recorded electoral alliances in Portugal, thought to illustrate the coding scheme and organization of data. 
Electoral alliance AP is formed by three parties, of which none is recorded in PCDB Party data (Table \ref{tab_party}) and thus \#\#999s are assigned. One party that forms electoral alliance PSP.US is identified as PSP; however it could not be validated how many parties form the alliance, and therefore the enumeraor is coded 99.
PDPC is knowingly formed by four parties of which only one (CDC) is recorded in the PCDB Party data.

Thought \texttt{\footnotesize pty\_eal\_id} often references \#\#999, it allows to link additional information on parties provided in Table \ref{tab_party} to the electoral-alliance information.

 Table \texttt{\footnotesize electoral\_alliances} is defined as follows:

\lstinputlisting[%caption={Code to compute the minimum-fragmentation Effective Number of Parties in Parliament.},%
language=postgreSQL]%
{../SQL-codes/tab_electoral_alliances.sql}


