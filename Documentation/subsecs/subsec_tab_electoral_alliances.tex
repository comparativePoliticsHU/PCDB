\subsection{Electoral Alliances}\label{subsec_tab_electoral_alliances}

Table \texttt{electoral\_alliances} provides information on electoral alliances, to identify the parties forming an electoral alliance when possible. 
Parties listed in the Party table (see \ref{subsec_tab_party}) that are recorded as electoral alliances are listed with their respective \texttt{pty\_id}.

Variable \texttt{pty\_eal\_nbr} is a counter that enumerates parties that constitute an electoral alliance.\footnote{The counter is also recorded in the Party table and equals one for all `conventional' parties.}
Accordingly, there occur as many rows for each electoral alliance in the table as variable \texttt{pty\_eal} counts. 

Variable \texttt{pty\_eal\_id}, in turn, records the party identifiers of the parties that form an electoral alliance. 
Combinations of \texttt{pty\_id} (electoral alliance) and \texttt{pty\_eal\_nbr} (enumerator of party in electoral alliance) are therefore unique.

\begin{table}[h!]
\centering\footnotesize
\caption{Example of composition of selected electoral alliances in Portugal.}\label{tab_electoral_alliance_example}
\begin{tabular}{c c *{1}{D{.}{.}{-1}} c c}
\tabularnewline\toprule\toprule
\multicolumn{3}{c}{Electoral Alliances} & \multicolumn{2}{c}{Party} \tabularnewline\addlinespace
\multicolumn{1}{c}{Identifier}	&	\multicolumn{1}{c}{Abbrevation}	&	\multicolumn{1}{c}{Enumerator}	&	\multicolumn{1}{c}{Identifier}	&	\multicolumn{1}{c}{Abbrevation}	\tabularnewline
\multicolumn{1}{c}{\texttt{\smallfont pty\_id}}	&	\multicolumn{1}{c}{\texttt{\smallfont pty\_abr}}	&	\multicolumn{1}{c}{\texttt{\smallfont pty\_eal\_n}}	&	\multicolumn{1}{c}{\texttt{\smallfont pty\_eal\_id}}	&	\multicolumn{1}{c}{ }	\tabularnewline
\midrule\addlinespace
8003	&	AP	&	1	&	8999	&	Other	\tabularnewline\addlinespace
8003	&	AP	&	2	&	8999	&	Other	\tabularnewline\addlinespace
8003	&	AP	&	3	&	8999	&	Other	\tabularnewline\addlinespace
8005	&	PSP.US	&	99	&	8058	&	PSP	\tabularnewline\addlinespace
8006	&	PDPC	&	1	&	8059	&	CDC	\tabularnewline\addlinespace
8006	&	PDPC	&	2	&	8999	&	Other	\tabularnewline\addlinespace
8006	&	PDPC	&	3	&	8999	&	Other	\tabularnewline\addlinespace
8006	&	PDPC	&	4	&	8999	&	Other	\tabularnewline\bottomrule\bottomrule\addlinespace
\end{tabular}
\end{table}

The example given in Table \ref{tab_electoral_alliance_example} presents a selection from the recorded electoral alliances in Portugal, and seeks to illustrate the coding scheme and organization of data in the table.
Electoral alliance AP is formed by three parties, of which none is recorded in PCDB Party data (see \ref{subsec_tab_party}) and thus \#\#999s are assigned. One party that forms electoral alliance PSP.US is identified as PSP; however it could not be validated how many parties form the alliance, and therefore the enumerator is coded 99.
The electoral alliance PDPC was knowingly formed by four parties, of which only one (CDC) is identified in the Party table.

Thought \texttt{pty\_eal\_id} often references \#\#999, it allows to link additional information on parties provided in table \ref{tab_party} to the electoral-alliance information.

\subparagraph{}
Table \texttt{electoral\_alliances} is defined as follows:
\lstinputlisting[%caption={Code to compute the minimum-fragmentation Effective Number of Parties in Parliament.},%
language=postgreSQL]%
{../SQL-codes/tables/tab_electoral_alliances.sql}


