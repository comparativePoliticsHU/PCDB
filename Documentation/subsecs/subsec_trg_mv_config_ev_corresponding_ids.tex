\subsection{Selecting corresponding institution identifiers within political configurations}\label{trg_mv_config_ev_prv_ids}
View Configuration Events (\ref{view_configuration_events}) sequences changes in the political-institutional configurations of a country by date.

Each row corresponds to a historically unique political configuration of government, lower house, upper house and the position of the Head of State. Political configurations are also uniquely identified by combinations of \texttt{\footnotesize ctr\_id} and \texttt{\footnotesize sdate}). 
The following excerpt illustrates what the structure of view Configuration Events looks like:\footnote{Poland has been chosen as an example because it is one of the few countries in the PCDB in which all political institutions of interest exist, as, besides lower and upper house, presidents are popularly elected since 1990.}

\begin{table}[h!]
\centering\footnotesize
\caption*{Example 3a: Excerpt from view Configuration Events with empty cells for temporally corresponding institution-configuratins.}
\begin{tabular}{r r r r r r}
\tabularnewline\toprule\toprule
\multicolumn{1}{r}{\texttt{\smallfont ctr\_id}}	&
\multicolumn{1}{r}{\texttt{\smallfont sdate}}	&	
\multicolumn{1}{r}{\texttt{\smallfont prselc\_id}}	&
\multicolumn{1}{r}{\texttt{\smallfont uh\_id}}	&
\multicolumn{1}{r}{\texttt{\smallfont lh\_id}}	&	
\multicolumn{1}{r}{\texttt{\smallfont cab\_id}}	\\\midrule
25	&	1993-09-19	&		&		&	25002	&		\\
25	&	1993-10-15	&		&	25002	&		&		\\
25	&	1993-10-26	&		&		&		&	25005	\\
25	&	1995-05-06	&		&		&		&	25006	\\
25	&	1995-12-23	&	25002	&		&		&		\\
25	&	1996-02-07	&		&		&		&	25007	\\
25	&	1997-09-21	&		&		&	25003	&		\\
25	&	1997-10-21	&		&	25003	&		&		\\
25	&	1997-11-11	&		&		&		&	25008	\\
25	&	2000-06-29	&		&		&		&	25009	\\
25	&	2000-12-23	&	25003	&		&		&		\\\bottomrule\bottomrule
\end{tabular}
\end{table}

Appraently, sequencing by start dates results in many empty cells. 
Yet, the second recorded president, who took office on December 23, 1995, was in charge throughout the subsequent five configurations. Thus, the presidential election identifier 25002 should also occur in this cells.
Particularly, computation of veto point in a given politcal configuration requires to fill the empty cells with the identifiers that refer to the cabinet, president, etc. that were in charge at a given point in time.

Because it is not possible to insert data into views, a materialized view that is identical with view Configuration Events is created: \texttt{\footnotesize mv\_configuration\_events}

To fill empty cells with temporally corresponding identifiers, a set of functions (\texttt{\footnotesize trg\_mv\_config\_ev\_prv\_*\_id()}) is created
Schematically, they are defined as follows:

\lstinputlisting[%caption={Code to create triggers that identify and insert corresponding  identifier in configuarations.},%
language=SQL]%
{../SQL-codes/trg_mv_config_ev_prv_ids_example.sql}

The function inserts the identifier of the institution-configuration that is currently in charge into empty cells, by choosing that one which was in charge in the previous political configuration.

Functions \texttt{\footnotesize trg\_mv\_config\_ev\_prv\_*\_id()} are triggered by insert, update, or delete from materialized view Configuration Events; events that occur when data in the base-tables is changed (see subsection \ref{integrity_and_consistency_of_MV}).


These procedures result in a structure that looks as follows:

\begin{table}[h!]
\centering\footnotesize
\caption*{Example 3b: Excerpt from materilaized view Configuration Events with cells of temporally corresponding institution-configuratins filled by triggers.}
\begin{tabular}{r r r r r r}
\tabularnewline\toprule\toprule
\multicolumn{1}{r}{\texttt{\smallfont ctr\_id}}	&
\multicolumn{1}{r}{\texttt{\smallfont sdate}}	&	
\multicolumn{1}{r}{\texttt{\smallfont prselc\_id}}	&
\multicolumn{1}{r}{\texttt{\smallfont uh\_id}}	&
\multicolumn{1}{r}{\texttt{\smallfont lh\_id}}	&	
\multicolumn{1}{r}{\texttt{\smallfont cab\_id}}	\\\midrule
25	&	1993-09-19	&	25001	&	25001	&	25002	&	25004	\\
25	&	1993-10-15	&	25001	&	25002	&	25002	&	25004	\\
25	&	1993-10-26	&	25001	&	25002	&	25002	&	25005	\\
25	&	1995-05-06	&	25001	&	25002	&	25002	&	25006	\\
25	&	1995-12-23	&	25002	&	25002	&	25002	&	25006	\\
25	&	1996-02-07	&	25002	&	25002	&	25002	&	25007	\\
25	&	1997-09-21	&	25002	&	25002	&	25003	&	25007	\\
25	&	1997-10-21	&	25002	&	25003	&	25003	&	25007	\\
25	&	1997-11-11	&	25002	&	25003	&	25003	&	25008	\\
25	&	2000-06-29	&	25002	&	25003	&	25003	&	25009	\\
25	&	2000-12-23	&	25003	&	25003	&	25003	&	25009	\\\bottomrule\bottomrule
\end{tabular}
\end{table}

The empty cells have been filled and materialized view Configuration Events can be used to compute the respective veto-potential configurations, cabinet seat shares in the lower and upper houses, and so forth.
