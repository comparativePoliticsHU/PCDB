\subsection{Parties}\label{subsec_tab_party}

Table \texttt{party} provides general information on parties, permitting to link them to other party-level databases or tables in the PCDB. 
Rows are parties within countries, identified by \texttt{pty\_id} or unique combinations of \texttt{ctr\_id} and \texttt{pty\_abr}.

\subparagraph{Party identifier}
The PCDB uses simple running counters to identify parties in a country's political system and history (variable \texttt{pty\_id}).
In contrast to the coding schemes applied in other political databases (e.g., \citealt{ManifestoData2013} or \citealt{ParlGov2012}), identifiers convey no meaning such as allignment with party-families or ideological leaning on a left-right scale.

Special suffix are assigned to
independent candidates (\#\#997), 
other parties with seats in the legislature (\#\#998), and 
other parties without seats in the legislature (\#\#999).
  
\subparagraph{}
Table \texttt{party} is defined as follows: 
\lstinputlisting[%caption={Code to compute the minimum-fragmentation Effective Number of Parties in Parliament.},%
language=postgreSQL]%
{../SQL-codes/tables/tab_party.sql}



