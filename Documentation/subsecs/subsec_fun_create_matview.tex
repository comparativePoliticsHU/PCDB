\subsection{Create materialized view}\label{subsec_fun_create_matview}

Function \texttt{create\_matview()} creates a table if not exists named as given by \texttt{matview\_name} as an exact copy of the view \texttt{view\_name}, and records its time of creation as time stamp in table Materialized Views (see \ref{subsec_tab_matviews}),
where \texttt{schema.matview\_name} and \texttt{schema.view\_name} are the first two non-optional input arguments.
The third argument takes the primary key column(s) as a quoted comma-sepaerated list, e.g. \texttt{'\{pkey\_col1, pkey\_col2\}'}.

Function \texttt{create\_matview()} is defined as follows:\footnote{Source is Listing 2 at \url{http://www.varlena.com/GeneralBits/Tidbits/matviews.html}.} 
\lstinputlisting[%caption={Code to create summary of institutions' start and election date differences.},%
language=postgreSQL,commentstyle=\color{white}]%
{../SQL-codes/functions/fun_create_matview.sql}

There also exist two functions which allow to refresh respectively drop a materialized view; definitions are provided in the Appendix (see \ref{subsec_appx_fun_refresh_matview} and \ref{subsec_appx_fun_refresh_matview}).
