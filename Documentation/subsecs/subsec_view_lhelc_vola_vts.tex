\subsection{Type A Volatility in Lower House Vote Shares}\label{view_lhelc_vola_vts}
View \texttt{\footnotesize view\_lhelc\_vola\_vts} is based on tables Lower House Election, Lower House Vote Results and Party, and provides data at the level of lower house elections.

Generally, type A volatility measures volatility from party entry and exit to the political system and is quantified by the change that occurs in the distribution of shares between parties due to parties newly entering respectively retiering from the electoral arena \citep{Powell&Tucker2013}, majorly the domestic party system or the lower house. 

Type A volatitlity in votes in a given lower house election is defined as volatility in the distribution of votes arising from new entering and retiering parties, given by the formular: 
\begin{equation}\label{equ_vote_a_volatility}
Vote\,A\,Volatility\,(k) = \frac{ | \sum\limits_{n=1}^{New} v_{n,k} + \sum\limits_{o=1}^{Old} v_{o,k} | }{2}
\end{equation}
, where $o$ refers to retiering parties that contested only the election $k-1$ and $n$ to new-entering
parties that contested only election $k$, and generally $v$ is party's vote share in the lower house election (i.e., the number of votes gained by party, divided by the total number of votes distributed between all parties $J$ that railed in the respective election $k$).


View \texttt{\footnotesize view\_lhelc\_vola\_vts} is programmed as follows:

\lstinputlisting[%caption={Code to compute type A volatitlity ins lower house votes.},%
language=SQL]%
{../SQL-codes/view_lhelc_vola_vts.sql}

Because the \texttt{\footnotesize SQL}-syntax of \texttt{\footnotesize view\_lhelc\_vola\_vts} is rather complex, some brief comments are instructive:
\begin{itemize}
\item[-]{The enumerator of Equ \ref{equ_vote_a_volatility} consists of two summands; each is computed seperately (the parts embraced by paranthesis and labeled \texttt{\footnotesize NEW\_PARTIES} and \texttt{\footnotesize LH\_RETIERING\_PARTIES}, respectively) and added only in the very first lines of the query.}
\item[-]{With respect to the subqueries, 
\texttt{\footnotesize NEW\_PARTIES} aggregates the vote shares of parties that contested in the present lower house election but not in the previous one, and 
 \texttt{\footnotesize LH\_RETIERING\_PARTIES} aggregates the vote shares of parties that contested in the previous election but not in the current one.}
\item[-]{Exluding `stable' parties (i.e., parties that entered the lower house in the present as well as the previous election) within the subqueries is achieved by the \texttt{\footnotesize EXCEPT}-clauses, which pair parties recorded for the present and the previous lower house by party identifiers. If a party contested only in the present election, or only in the previous elections, then it does not occur in the query that follows the \texttt{\footnotesize EXCEPT}-clauses. In consequence, only votes gained by new entering and retiering parties enter the aggregation.}
\item[-]{The category `Others without seat' (\texttt{\footnotesize pty\_id} is \#\#999) are excluded from the computation of individual parties' vote shares, because volatility in the lower house is of interest (not volatility in the party system more generally).}
\item[-]{Generally, joining parties' vote results with different combinations of the identifiers of the previous, the current, and the next lower house election enables to easily identify new entering and retiering parties.}
\end{itemize}

{\bf Note}:	Figures for first an last recorded elections are unreliable, because it is impossible to determine which parties are 'newcomers' in first and which parties will retier in last election, respectively. 
%The exclusion of first lower house configurations is a consequence of the final left-outer join, as the first lower house reported by subquery \texttt{\footnotesize NEW\_PARTIES} is always only the second for a given country. 
%The excludsion of first lower house configurations, in turn, is achieved by selecting only parties vote results from lower house elections that have not the highest within country election identifier.\footnote{specifically, that is quering \texttt{\smallfont \ldots SELECT lhelc\_id, pty\_id, pty\_lh\_sts \ldots WHERE lhelc\_id NOT IN (SELECT MAX(lhelc\_id) \ldots}. A feasible alternative would be programming the restriction based on selection of the election with the highest date within a country (would prevent from coding failures in the ordering of lower house election identifiers).}



