\subsection{Effective Number of Parties in Parliament, Minimum Fragmentation}\label{subsec_view_lh_enpp_minfrag}
View \texttt{view\_lh\_enpp\_minfrag} is based on table Lower House Seat Results, and aggregates data at the level of lower houses.

The effective number of parties in parliament (ENPP) is a measure of party system fractionalization that takes into acount the relative size of parties present in a country's lower house. 
%In addition to an recorded figure (Formular \ref{ENPP_equ}), the PCDB provides for two ENPP indices that are computed based on the recorded lower house seat and vote results.

Variable \texttt{lh\_enpp\_minfrag} is computed based on the formula originally proposed by \citet{Laakso&Taagepera1979}
\begin{equation}\label{ENPP_equ_minfrag}
\mbox{ENPP}_{\mbox{\tiny minfrag}}(k) = 1/\sum\limits_{j=1}^{J}s_{j,k}^{2},
\end{equation}
where $k$ denotes a country's lower house at a given point in time, $J$ are parties in a given lower house $k$, and $s$ is party $j$'s seat share in the $k$th lower house. 

View \texttt{view\_lh\_enpp\_minfrag} is defined as follows:
\lstinputlisting[%caption={Code to compute the minimum-fragmentation Effective Number of Parties in Parliament.},%
language=SQL]%
{../SQL-codes/views/view_lh_enpp_minfrag.sql}
Note that the ENPP is calculated with the computed, not the recorded total number of parties' seats in the lower house.

The variable suffix \texttt{\_minfrag} points to the fact that \citeauthor{Laakso&Taagepera1979}'s original formula lumps small parties or independent representatives in the parliaement into one single categories (here the categories `Others with seats' [\texttt{otherw}] and `Independents' [\texttt{IND}]). 
other parties with seats and independents, respectively, enter into the calculation as if they each form a single party, and thus tend to increase the fractionalization indice only marignally. 
Hence, this is equivalent to assume minimum fragmentation, and this likley results in an underestimate of fragmentation \citep[cf.][]{Gallagher&Mitchell2005}.

The PCDB provides for an alternative ENPP indice that adjusts for this tendency (see \ref{subsec_view_lh_enpp_maxfrag}).



