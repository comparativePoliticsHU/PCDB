\subsection{Type B Volatility in Lower House Seat Shares}\label{subsec_view_lh_volb_sts}

View \texttt{view\_lhelc\_volb\_sts} is based on tables Lower House and Lower House Seat Results, and provides data at the level of lower houses.

Type B volatility quantifies the change that occurs in the distribution of seat shares within parties in subseqent lower houses, comparing the results in the current to that of the previous one. 
Accordingly, type B volatility considers only so-called stable parties and measures the  volatility in the distribution of seats arising from gaines and losses of these stable parties.	

The formula to compute \texttt{lh\_volb\_sts} is
\begin{equation}
\mbox{Seat\,B\,Volatility}(k) = \frac{ \bigg| \sum\limits_{j=1}^{Stable} s_{j,(k-1)} - s_{j,k}\bigg| }{2},
\end{equation}
where $s$ are seat or vote shares that party $j$ gained in the current lower house $k$ or in the previous lower house $k-1$.

View \texttt{view\_lh\_volb\_sts} is defined as follows:
\lstinputlisting[%caption={Code to compute type B volatitlity ins lower house seats.},%
language=SQL]%
{../SQL-codes/views/view_lh_volb_sts.sql}

Stable parties are identified computationable by calculating the cross-product between rows in the subqueries 
\texttt{CUR\_LH} and \texttt{PREV\_LH}, and reporting only those for which a party identifier is enlisted in both the previous and the current election.% (cf. corresponding \texttt{WHERE}-clause). 

Note that the stable partis cannot be identified for every first recorded lower house, and hence B volaities are missing for these institutional configurations.
It may be also worth highlighting that indicator is highly sensitive to missing data in the tables it references, as no aggregate value is computed for lower house elections in which at least one party except the group `Others withour seat' has \texttt{NULL} records for both seats gained by plurality and proportional vote.
A lack of reliable lower-level data thus causes missingness at the aggregate level. 
% 
% Generally, consistency check \texttt{cc\_lh\_volb\_sts} [\ref{cc_lh_volb_sts}]) provides for a comparison of the computed and the recorded figures, though the recorded have been computed manually as well.

