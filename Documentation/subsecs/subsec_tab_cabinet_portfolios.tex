\subsection{Cabinet Portfolios}\label{subsec_tab_cabinet_portfolios}
Table \texttt{cabinet\_portfolios} provides information on parties in cabinets. 

As cabinet portfolio we define the composition of a cabinet at the party-level.
%, i.e., the parties that form the, government, the number of a party’s share on total cabinet seats, and parties supporting government. 
Thus, new portfolios are included whenever a new cabinet emerges.
The changes that occur at the party-level regularly correspond to the events enumerated as criteria for recording a new cabinet configuration (cf. subsection \ref{subsec_tab_cabinet}):
\begin{itemize}%\itemsep-4pt \parsep0pt
\item[a)] Coalition composition changes.
\item[b)] Head of government changes.
\item[c)] Government formation after general legislative elections (not in presidential systems).
\end{itemize}
Obviously, combinations of cabinet and party identifier are unique in the cabinet portfolios table.

Table \texttt{cabinet\_portfolios} is defined as follows:
\lstinputlisting[%caption={Code to compute the minimum-fragmentation Effective Number of Parties in Parliament.},%
language=postgreSQL]%
{../SQL-codes/tables/tab_cabinet_portfolios.sql}
