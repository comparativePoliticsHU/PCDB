\subsection{Lower House Elections}\label{subsec_tab_lh_election}

Table \texttt{lh\_election} provides information on lower house elections. 
Rows are lower house elections, identified by \texttt{lhelc\_id}. It is noteworthy that each lower house election corresponds to a lower house configuration (cf. subsection \ref{subsec_tab_lower_house}).\footnote{%
While the opposite, that each lower house configuration corresponds to a lower house election, is not true.}

\subparagraph{Elections, pluarality versus proportional voting, and seat allocation}
Lower house election dates (\texttt{lhelc\_date}), and figures on registered voters (\texttt{lhelc\_reg\_vts\*}), the number valid votes (\texttt{lhelc\_vts\_\*}), and the number of seats elected (\texttt{lhelc\_sts\_*}) are recorded in accordance with official statistics, if available. 
Else, \citet{Nohlen2001, Nohlen2005, Nohlen2010} is the primary source, complemented by individual-case research. Information on data sources is provided in variable \texttt{lhelc\_src}.

\subparagraph{Electoral system}
Key information on the electoral system to elect the lower house is provided for each tier disaggregatedly namely
\begin{itemize}%\itemsep-4pt \parsep0pt
\item[-]{the electoral formular (\texttt{lhelc\_fml\_t*}), as defined by a customed type \texttt{elec\_formula}
%\footnote{The PCDB distinguishes between the following electural formular: Two Round System (2RS), Alternative Vote (AV), DHondt, Droop, Droop with Largest-Remainders (LR-Droop), Hare, modified Hare, Hare with Largest-Remainders (LR-Hare), Highest Average Remaining, Imperiali, Multi-Member District (MMD), mSainteLague, Reinforced Imperiali, SainteLague, Single Member Plurality (SMP), Single Non-Transferable Vote (SNTV), and Single Transferable Vote (STV).%}
,}
\item[-]{the number of constituencies (\texttt{lhelc\_ncst\_t*}),}
\item[-]{the number of seats allocated(\texttt{lhelc\_sts\_t*}),}
\item[-]{the average district magnitude (\texttt{lhelc\_mag\_t*}),}
\item[-]{the national threshold (\texttt{lhelc\_ntrsh\_t*}), and}
\item[-]{the district threshold (\texttt{lhelc\_dtrsh\_t*}).}
\end{itemize}

Type \texttt{elec\_formula} is defined as follows:

\lstinputlisting[%caption={Code to compute the minimum-fragmentation Effective Number of Parties in Parliament.},%
language=postgreSQL]%
{../SQL-codes/types/type_elec_formula.sql}

In addition, 
%\subparagraph{Mean and median average district magnitude}
variables \texttt{lhelc\_dstr\_mag} and \texttt{lhelc\_dstr\_mag\_med} aggregate the average district magnitudes across the different tiers of the electoral system, reporting the mean and the median, respectively.

Comments and information on the sources of data on the electoral system are provided in \texttt{lhelc\_esys\_cmt} and \texttt{lhelc\_esys\_src}, respectively.

\subparagraph{}
Table \texttt{lh\_election}  is defined as follows:
\lstinputlisting[%caption={Code to compute the minimum-fragmentation Effective Number of Parties in Parliament.},%
language=postgreSQL]%
{../SQL-codes/tables/tab_lh_election.sql}
