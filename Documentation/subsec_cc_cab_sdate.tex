\subsection{CC Cabinet start date}\label{cc_cab_sdate}
Consistency check \texttt{\footnotesize cc\_cab\_sdate} is based on the Cabinet and Lower House Election tables and provides information at the level of cabinet configurations.

It compares a cabinet's start date (\texttt{\footnotesize cab\_sdate}) and the date of the lower house election it originates from (\texttt{\footnotesize lhelc\_date}). Variable \texttt{\footnotesize date\_dif} measures the difference between both dates in days; variable \texttt{\footnotesize prob\_corr\_ddif} is zero when the recorded date of cabinet formation, i.e., its start date, and the election date are equal. 

CC \texttt{\footnotesize cc\_cab\_sdate} is programmed as follows:

\lstinputlisting[%caption={Code to create consistency check on cabinet start dates.},%
language=SQL]%
{../SQL-codes/cc_cab_sdate.sql}


The explicit, and actually empirically reasonable assumption is that government formation regularly takes some days in the countries that are covered in the PCDB. Because in the coding process the date of the election a cabinet originates from has been used as default cabinet start date, if the difference is equal to zero, this strongly indicates that no proper start date has been recorded. Case-specific research is still required for these cabient configurations.

\emph{\textbf{Note}}: Information on the sources of cabinet start dates is stored in variable \texttt{\footnotesize cab\_src} of the Cabinet table and variable \texttt{\footnotesize valid\_cab\_sdate}, in turn, is an individually coded dummy that indicates whether individual case research has been properly conducted an the source of information appears reliable.

