\subsection{Effective Number of Parties in Parliament, Minimum Fragmentation}\label{view_lh_enpp_minfrag}
View \texttt{\footnotesize view\_lh\_enpp\_minfrag} is based on table Lower House and Lower House Seat Results, and provides data at the level of lower houses.

The effective number of parties in parliament (ENPP) is a measure of party system fractionalization that takes into acount the relative size of parties present in a country's lower house. 
%In addition to an recorded figure (Formular \ref{ENPP_equ}), the PCDB provides for two ENPP indices that are computed based on the recorded lower house seat and vote results.

Variable \texttt{\footnotesize lh\_enpp\_minfrag} is computed based on the formula originally proposed by \citet{Laakso&Taagepera1979}
\begin{equation}\label{ENPP_equ_minfrag}
ENPP_{minfrag}(k) = 1/\sum\limits_{j=1}^{J}s_{j,k}^{2}
\end{equation}
, where $k$ denotes a country's lower house at a given point in time, $J$ are parties in a given lower house $k$, and $s$ is party $j$'s seat share in the $k$th lower house. 

The suffix \texttt{\footnotesize \_minfrag} points to the fact that Laakso \& Taagepera's original formular lumps small parties or single representatives in the parliement into single categories (here the categories `Others with seats' [otherw] and `Independents' [IND]). 
This is equivalent to assume minimum fragmentation, other parties and independents enter into the calculation as if it were a single party, and thus tend to increase the fractionalization indice only marignally. Apparantly, this likley results in an underestimate of fragmentation \citep{Gallagher&Mitchell2005}.

The PCDB provides for an alternative ENPP indice that adjusts for this tendency.

View \texttt{\footnotesize view\_lh\_enpp\_minfrag} is programmed as follows:

\lstinputlisting[%caption={Code to compute the minimum-fragmentation Effective Number of Parties in Parliament.},%
language=SQL]%
{../SQL-codes/view_lh_enpp_minfrag.sql}

{\bf Note}: Computation of the ENPPs is proceeded with the computed, not the recorded total number of Lower House seats (note that the computed sum of seats might deviate from the recorded figure in table Lower House; cf. \texttt{\footnotesize cc\_lhelc\_sts\_ttl}).



