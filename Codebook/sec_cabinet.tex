\section{Cabinet}\label{sec_cabinet}
This table contains information on cabinets.
Rows are the different cabinet configurations, identified by variable \texttt{\footnotesize cab\_id}. 
A new cabinet is enlisted if one of the following events took place:
\begin{itemize}\label{cabinet_change_criteria}\itemsep-4pt \parsep0pt
\item[a)] Coalition composition changes at the party-level.
\item[b)] Head of government changes.
\item[c)] Government formation after general legislative elections (not in presidential systems).
\end{itemize}

\subparagraph{Cabinet start date} Variable \texttt{\footnotesize cab\_sdate} refers to the date on which the cabinet, as proposed by the Head of Government, recieves a vote of confidence in the legislature. The variable \texttt{\footnotesize cab\_src} regularly contains links to the websites or online repositories which are used as references. If available, data was compiled directly from information reported on government websites or other official sources.

\subparagraph{Total number of cabinet portfolios} In the present version of the database (!) the number of cabinet portolios is an integer counter equal to the number of parties in cabinet, as listed in table \ref{tab_cabinet_portfolios}. 

\subparagraph{Sources} Information is obtained from \citet*{Woldendrop_et_al2000} and the Political Data Yearbook \citeyearpar{EJPR_PDY}, and was complemented by individual-case research.

%The table contains the following variables:

\begin{center}
[table Variables in Cabinet Table on next page]
\end{center}

\newpage
\begin{center}
\begin{longtable}{L{4cm} L{6cm} L{3cm}}
\caption{Variables in Cabinet Table\label{tab_cabinet}}
\tabularnewline\addlinespace

%& &\tabularnewline
\toprule\toprule
\emph{\textbf{Variable}}        &       \emph{\textbf{Description}}     &       \emph{\textbf{Format}}  \tabularnewline
\midrule
\endfirsthead

\caption[]{\emph{\ldots\ continued}}\tabularnewline\addlinespace
%& &\tabularnewline
\toprule\toprule
\emph{\textbf{Variable}}        &       \emph{\textbf{Description}}     &       \emph{\textbf{Format}}  \tabularnewline
\midrule
\endhead



\addlinespace
\multicolumn{3}{c}{{\emph{continued on next page \ldots}}} \tabularnewline\addlinespace%\bottomrule
\endfoot

\bottomrule\bottomrule
\endlastfoot
cab\_id 	&	       Cabinet identifier      	&	Numeric(5,0)	\tabularnewline\addlinespace
cab\_prv\_id    	&	       Cabinet identifier of the previous cabinet      	&	Numeric(5,0)	\tabularnewline\addlinespace
ctr\_id 	&	       Country identifier      	&	Integer	\tabularnewline\addlinespace
cab\_sdate      	&	       Cabinet start date      	&	YYYY-MM-DD	\tabularnewline\addlinespace
cab\_hog\_n     	&	       Name of the Head of Government  	&	Character	\tabularnewline\addlinespace
cab\_sts\_ttl   	&	       Total number of cabinet portfolios      	&	Numeric	\tabularnewline\addlinespace
cab\_care       	&	       Indicates if cabinet is a caretaker cabinet     	&	Boolean	\tabularnewline\addlinespace
cab\_cmt        	&	       Comments        	&	Text	\tabularnewline\addlinespace
cab\_src        	&	       Data sources    	&	Text	\tabularnewline\addlinespace
cab\_valid\_sdate & Indicates whether cabinet start date has been double-checked & Boolean \tabularnewline\addlinespace
\end{longtable}
\end{center}
