\chapter{Tables}\label{chap_tables}

The tables in the PCDB record the primary data on countries' political institutions, parties, elections, and electoral systems.

Each table provides data on exclusively one institution and/or type of event.\footnote{This is thought to avoid redundancy.} 
\emph{Prefixes of variable names} in the tables are often abbrevations of types of institutions or instutional events (e.g, \texttt{\footnotesize cab} for variables containing information on cabinets, \texttt{\footnotesize lhelc} for variables containing information on lower house elections, etc.). 

\emph{Rows} in tables are unique data points with regard to the configuration of interest (e.g., historically distinct lower house configurations). Criteria for what constitutes a unique data point is provided in the introductions to the respective sections. 

\emph{Columns} represent the variables contained in a table. 
The first column of a table is usually an identifier, indicated by the suffix \texttt{\footnotesize \_id}, which is principally a sequencial counter that is unique within countries.

\subparagraph{Technical note}
Note that tables usually store the primary information and data contained in the database, whereas views report aggregate data, such as totals or computed indices.
However, much of the primary data provided in official statistics is already aggregated (e.g., total votes or vote turnout at the national level). 
These figures are recorded in the tables according to primary sources.
Computation of indices is regularly proceeded with primary data at the lowest conceptional level. Detailed information on computed variables in views is provided in Chapter \ref{Chapter_Views}.
 