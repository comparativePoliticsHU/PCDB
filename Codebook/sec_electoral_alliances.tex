\section{Electoral Alliances}\label{sec_electoral_alliances}
This table provides information on electoral alliances, attempting to identify the parties forming an electoral alliance. Parties listed in Table \ref{tab_party} that are recorded as electoral alliances are listed in Table \ref{tab_electoral_alliances} with their respective \texttt{\footnotesize pty\_id}.

Variable \texttt{\footnotesize pty\_eal\_nbr} is a counter that enumerates parties that constitute an electoral alliance.\footnote{The counter is also recorded in Table \ref{tab_party} and equals one for all `conventional' parties.}
Accordingly, there occur as many rows for each electoral alliance in Table \ref{tab_electoral_alliances} as variable \texttt{\footnotesize pty\_eal} counts. 

Variable \texttt{\footnotesize pty\_eal\_id}, in turn, records the party identifiers of the parties that form an electoral alliance. Combinations of \texttt{\footnotesize pty\_id} (electoral alliance) and \texttt{\footnotesize pty\_eal\_nbr} (enumerator of party in electoral alliance) are therefore unique within countries.

\begin{table}[h!]
\centering\footnotesize
\caption*{Example 1: Composition of selected electoral alliances in Portugal.}
\begin{tabular}{c c *{1}{D{.}{.}{-1}} c c}
\tabularnewline\toprule\toprule
\multicolumn{3}{c}{Electoral Alliances} & \multicolumn{2}{c}{Party} \tabularnewline\addlinespace
\multicolumn{1}{c}{Identifier}	&	\multicolumn{1}{c}{Abbrevation}	&	\multicolumn{1}{c}{Enumerator}	&	\multicolumn{1}{c}{Identifier}	&	\multicolumn{1}{c}{Abbrevation}	\tabularnewline
\multicolumn{1}{c}{\texttt{\smallfont pty\_id}}	&	\multicolumn{1}{c}{\texttt{\smallfont pty\_abr}}	&	\multicolumn{1}{c}{\texttt{\smallfont pty\_eal\_n}}	&	\multicolumn{1}{c}{\texttt{\smallfont pty\_eal\_id}}	&	\multicolumn{1}{c}{ }	\tabularnewline
\midrule\addlinespace
8003	&	AP	&	1	&	8999	&	Other	\tabularnewline\addlinespace
8003	&	AP	&	2	&	8999	&	Other	\tabularnewline\addlinespace
8003	&	AP	&	3	&	8999	&	Other	\tabularnewline\addlinespace
8005	&	PSP.US	&	99	&	8058	&	PSP	\tabularnewline\addlinespace
8006	&	PDPC	&	1	&	8059	&	CDC	\tabularnewline\addlinespace
8006	&	PDPC	&	2	&	8999	&	Other	\tabularnewline\addlinespace
8006	&	PDPC	&	3	&	8999	&	Other	\tabularnewline\addlinespace
8006	&	PDPC	&	4	&	8999	&	Other	\tabularnewline\bottomrule\bottomrule\addlinespace
\end{tabular}
\end{table}

Example 1 displays a selection from the recorded electoral alliances in Portugal, thought to illustrate the coding scheme and organization of data. 
Electoral alliance AP is formed by three parties, of which none is recorded in PCDB Party data (Table \ref{tab_party}) and thus \#\#999s are assigned. One party that forms electoral alliance PSP.US is identified as PSP; however it could not be validated how many parties form the alliance, and therefore the enumeraor is coded 99.
PDPC is knowingly formed by four parties of which only one (CDC) is recorded in the PCDB Party data.

Thought \texttt{\footnotesize pty\_eal\_id} often references \#\#999, it allows to link additional information on parties provided in Table \ref{tab_party} to the electoral-alliance information. 

\begin{center}
\begin{longtable}{L{4cm} L{6cm} L{3cm}}
\caption{Variables in Electoral Alliances Table\label{tab_electoral_alliances}}
\tabularnewline\addlinespace

%& &\tabularnewline
\toprule\toprule
\emph{\textbf{Variable}}        &       \emph{\textbf{Description}}     &       \emph{\textbf{Format}}  \tabularnewline
\midrule
\endfirsthead

\caption[]{\emph{\ldots\ continued}}\tabularnewline\addlinespace
%& &\tabularnewline
\toprule\toprule
\emph{\textbf{Variable}}        &       \emph{\textbf{Description}}     &       \emph{\textbf{Format}}  \tabularnewline
\midrule
\endhead



\addlinespace
\multicolumn{3}{c}{{\emph{continued on next page \ldots}}} \tabularnewline\addlinespace%\bottomrule
\endfoot

\bottomrule\bottomrule
\endlastfoot
ctr\_id	&	Country identifier	&	Integer	\tabularnewline\addlinespace
pty\_id	&	Party identifier	&	Numeric(5,0)	\tabularnewline\addlinespace
pty\_abr	&	Party abbrevation	&	Character	\tabularnewline\addlinespace
pty\_eal	\_nbr &	Indicates the number of parties participating in an electoral alliance	&	Integer	\tabularnewline\addlinespace
pty\_eal\_id	&	Electoral alliance party identifier	&	Numeric(5,0)	\tabularnewline\addlinespace
pty\_eal\_cmt	&	Comment	&	Text	\tabularnewline\addlinespace
pty\_eal\_src	&	Source of inforamtion on party's participation in electoral alliance	&	Text	\tabularnewline\addlinespace
\end{longtable}
\end{center}
