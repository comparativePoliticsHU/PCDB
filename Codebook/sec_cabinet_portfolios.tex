\section{Cabinet Portfolios}\label{sec_cabinet_portfolios}
This table provides information on cabinet portfolios. 

As cabinet portfolio we define the composition of a cabinet at the party-level.
%, i.e., the parties that form the, government, the number of a party’s share on total cabinet seats, and parties supporting government. 
Thus, new portfolios are included whenever a new cabinet emerges.
The changes that occur at the party-level regularly correspond to the events enumerated as criteria for recording a new cabinet configuration (cf. Section \ref{sec_cabinet}):
\begin{itemize}%\itemsep-4pt \parsep0pt
\item[a)] Coalition composition changes.
\item[b)] Head of government changes.
\item[c)] Government formation after general legislative elections (not in presidential systems).
\end{itemize}
Obviously, combinations of cabinet and party identifier are unique in the cabinet portfolios table.

Information is obtained from \citet*{Woldendrop_et_al2000} and the Political Data Yearbook \citeyearpar{EJPR_PDY}, and was complemented by individual-case research.

%The table contains the following variables:


\begin{center}
\begin{longtable}{L{4cm} L{6cm} L{3cm}}
\caption{Variables in Cabinet Portfolios Table\label{tab_cabinet_portfolios}}
\tabularnewline\addlinespace

%& &\tabularnewline
\toprule\toprule
\emph{\textbf{Variable}}        &       \emph{\textbf{Description}}     &       \emph{\textbf{Format}}  \tabularnewline
\midrule
\endfirsthead

\caption[]{\emph{\ldots\ continued}}\tabularnewline\addlinespace
%& &\tabularnewline
\toprule\toprule
\emph{\textbf{Variable}}        &       \emph{\textbf{Description}}     &       \emph{\textbf{Format}}  \tabularnewline
\midrule
\endhead



\addlinespace
\multicolumn{3}{c}{{\emph{continued on next page \ldots}}} \tabularnewline\addlinespace%\bottomrule
\endfoot

\bottomrule\bottomrule
\endlastfoot
ptf\_id 	&	       Portfolio identifier    	&	Numeric(5,0)	\tabularnewline\addlinespace
cab\_id 	&	       Cabinet identifier      	&	Numeric(5,0)	\tabularnewline\addlinespace
pty\_id 	&	       Party identifier        	&	Numeric(5,0)	\tabularnewline\addlinespace
pty\_cab        	&	       Indicates if party is in cabinet        	&	Boolean	\tabularnewline\addlinespace
pty\_cab\_sts   	&	       A party’s number of portfolios/ministries in a cabinet  	&	Numeric	\tabularnewline\addlinespace
pty\_cab\_hog   	&	       Indicates if party fills the position of the Head of Government 	&	Boolean	\tabularnewline\addlinespace
pty\_cab\_sup   	&	       Indicates if party is supporting the cabinet but is not part of it      	&	Boolean	\tabularnewline\addlinespace
ptf\_cmt        	&	       Comments        	&	Text	\tabularnewline\addlinespace
ptf\_src        	&	       Data sources    	&	Text	\tabularnewline\addlinespace
\end{longtable}
\end{center}


