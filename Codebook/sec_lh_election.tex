\section{Lower House Election}\label{sec_lh_election}
This table provides information on lower house elections. Rows are lower house elections, identified by \texttt{\footnotesize lhelc\_id}. It is noteworthy that each lower house election corresponds to a lower house configuration (cf. Section \ref{sec_lower_house}).\footnote{While the opposite, that each lower house configuration corresponds to a lower house election, is not true.}

\subparagraph{Elections, pluarality versus proportional voting, and seat allocation}
Lower house election dates (\texttt{\footnotesize lhelc\_date}), and figures on registered voters (\texttt{\footnotesize lhelc\_reg\_vts*}), the number valid votes (\texttt{\footnotesize lhelc\_vts\_*}), and the number of seats elected (\texttt{\footnotesize lhelc\_sts\_*}) are recorded in accordance with official statistics, if available. 
Else, \citet{Nohlen2001, Nohlen2005, Nohlen2010} is the primary source, complemented by individual-case research. Information on data sources is provided in variable \texttt{\footnotesize lhelc\_src}.

\subparagraph{Electoral system}
Key information on the electoral system to elect the lower house is provided for each tier disaggregatedly namely
\begin{itemize}\itemsep-4pt \parsep0pt
\item[-]{the electoral formular\footnote{%
The PCDB distinguishes between the following electural formular: Two Round System (2RS), Alternative Vote (AV), DHondt, Droop, Droop with Largest-Remainders (LR-Droop), Hare, modified Hare, Hare with Largest-Remainders (LR-Hare), Highest Average Remaining, Imperiali, Multi-Member District (MMD), mSainteLague, Reinforced Imperiali, SainteLague, Single Member Plurality (SMP), Single Non-Transferable Vote (SNTV), and Single Transferable Vote (STV).%
} (\texttt{\footnotesize lhelc\_fml\_t*}),}
\item[-]{the number of constituencies (\texttt{\footnotesize lhelc\_ncst\_t*}),}
\item[-]{the number of seats allocated(\texttt{\footnotesize lhelc\_sts\_t*}),}
\item[-]{the average district magnitude (\texttt{\footnotesize lhelc\_mag\_t*}),}
\item[-]{the national threshold (\texttt{\footnotesize lhelc\_ntrsh\_t*}), and}
\item[-]{the district threshold (\texttt{\footnotesize lhelc\_dtrsh\_t*}).}
\end{itemize}
In addition, 
%\subparagraph{Mean and median average district magnitude}
variables \texttt{\footnotesize lhelc\_dstr\_mag} and \texttt{\footnotesize lhelc\_dstr\_mag\_med} aggregate the average district magnitudes across the different tiers of the electoral system, reporting the mean and the median, respectively.

Comments and information on the sources of data on the electoral system are provided in \texttt{\footnotesize lhelc\_esys\_cmt} and \texttt{\footnotesize lhelc\_esys\_src}, respectively.

\subparagraph{Sources} Information is obtained  from \citet{Nohlen2001, Nohlen2005, Nohlen2010}, and complemented by individual-case research.

%The table contains the following variables:
%\newpage
%\begin{center}
%[table Variables in Lower House Election Table on next page]
%\end{center}
%\newpage

\begin{center}%ing
\begin{longtable}{L{4cm} L{6cm} L{3cm}}%\hline
\caption{Variables in Lower House Election Table}\label{tab_lh_election}
\tabularnewline\addlinespace

%& &\tabularnewline
\toprule\toprule
\emph{\textbf{Variable}}        &       \emph{\textbf{Description}}     &       \emph{\textbf{Format}}  \tabularnewline
\midrule
\endfirsthead

\caption[]{\emph{\ldots\ continued}}\tabularnewline\addlinespace
%& &\tabularnewline
\toprule\toprule
\emph{\textbf{Variable}}        &       \emph{\textbf{Description}}     &       \emph{\textbf{Format}}  \tabularnewline
\midrule
\endhead



\addlinespace
\multicolumn{3}{c}{{\emph{continued on next page \ldots}}} \tabularnewline\addlinespace%\bottomrule
\endfoot

\bottomrule\bottomrule
\endlastfoot
lhelc\_id       	&	       Lower house election identifier 	&	Numeric(5,0)	\tabularnewline\addlinespace
lhelc\_prv\_id  	&	       Previous lower house election identifier        	&	Numeric(5,0)	\tabularnewline\addlinespace
ctr\_id 	&	       Country identifier      	&	Integer	\tabularnewline\addlinespace
lhelc\_date     	&	       Lower house election date       	&	YYYY-MM-DD	\tabularnewline\addlinespace
lhelc\_early    	&	       Indicates an early election     	&	Boolean	\tabularnewline\addlinespace
lhelc\_reg\_vts 	&	       Number of registered voters     	&	Numeric	\tabularnewline\addlinespace
lhelc\_reg\_vts\_pr     	&	       Number of registered voters, PR system  	&	Numeric	\tabularnewline\addlinespace
lhelc\_reg\_vts\_pl     	&	       Number of registered voters, plurality system   	&	Numeric	\tabularnewline\addlinespace
lhelc\_vts\_pr  	&	       Valid votes for lower house elected with proportional representation system     	&	Numeric	\tabularnewline\addlinespace
lhelc\_vts\_pl  	&	       Valid votes for lower house elected with plurality system        	&	Numeric	\tabularnewline\addlinespace
lhelc\_sts\_pr  	&	       Number of lower house seats elected with proportional representation system     	&	Numeric	\tabularnewline\addlinespace
lhelc\_sts\_pl  	&	       Number of lower house seats elected with plurality system       	&	Numeric	\tabularnewline\addlinespace
lhelc\_sts\_ttl 	&	       Total number of lower house seats elected in the election       	&	Numeric	\tabularnewline\addlinespace
lhelc\_fml\_t1  	&	       Electoral formula used for allocation of lower house seats on the first tier     	&	Character	\tabularnewline\addlinespace
lhelc\_ncst\_t1 	&	       Number of lower house constituencies at the first tier  	&	Numeric	\tabularnewline\addlinespace
lhelc\_sts\_t1  	&	       Number of lower house seats allocated at the first tier 	&	Numeric	\tabularnewline\addlinespace
lhelc\_dstr\_mag        	&	 Mean average lower house district magnitude\footnote{Data obtained from \citet*{Carey&Hix2011}.} 	&	Numeric	\tabularnewline\addlinespace
lhelc\_dstr\_mag\_med   	&	 Median average lower house district magnitude\footnote{Data and definition provided by\citet*{Carey&Hix2008}.}	&	Numeric	\tabularnewline\addlinespace
lhelc\_ mag\_t1 	&	       Average lower house district magnitude on first tier    	&	Numeric	\tabularnewline\addlinespace
lhelc\_ntrsh \_t1       	&	       National threshold for lower house on the first tier    	&	Numeric	\tabularnewline\addlinespace
lhelc\_dtrsh\_t1        	&	       District threshold for lower house on first tier        	&	Numeric	\tabularnewline\addlinespace
lhelc\_fml\_t2  	&	       Electoral formula used for allocation of lower house seats on the second  tier       	&	Character	\tabularnewline\addlinespace
lhelc\_ncst\_t2 	&	       Number of lower house constituencies at the second tier 	&	Numeric	\tabularnewline\addlinespace
lhelc\_sts\_t2  	&	       Number of lower house seats allocated at the second tier        	&	Numeric	\tabularnewline\addlinespace
lhelc\_ mag\_t2 	&	       Average lower house district magnitude on second tier   	&	Numeric	\tabularnewline\addlinespace
lhelc\_ntrsh \_t2       	&	       National threshold for lower house on the second tier   	&	Numeric	\tabularnewline\addlinespace
lhelc\_dtrsh\_t2        	&	       District threshold for lower house on second tier       	&	Numeric	\tabularnewline\addlinespace
lhelc\_fml\_t3  	&	       Electoral formula used for allocation of lower house seats on the third tier 	&	Character	\tabularnewline\addlinespace
lhelc\_ncst\_t3 	&	       Number of lower house constituencies at the third tier  	&	Numeric	\tabularnewline\addlinespace
lhelc\_sts\_t3  	&	       Number of lower house seats allocated at the third tier 	&	Numeric	\tabularnewline\addlinespace
lhelc\_ mag\_t3 	&	       Average lower house district magnitude on third tier    	&	Numeric	\tabularnewline\addlinespace
lhelc\_ntrsh \_t3       	&	       national threshold for lower house on the third tier    	&	Numeric	\tabularnewline\addlinespace
lhelc\_dtrsh\_t3        	&	       District threshold for lower house on third tier        	&	Numeric	\tabularnewline\addlinespace
lhelc\_fml\_t4  	&	       Electoral formula used for allocation of lower house seats on the fourth tier        	&	Character	\tabularnewline\addlinespace
lhelc\_ncst\_t4 	&	       Number of lower house constituencies at the fourth tier 	&	Numeric	\tabularnewline\addlinespace
lhelc\_sts\_t4  	&	       Number of lower house seats allocated at the fourth tier        	&	Numeric	\tabularnewline\addlinespace
lhelc\_mag\_t4  	&	       Average lower house district magnitude on fourth tier   	&	Numeric	\tabularnewline\addlinespace
lhelc\_ntrsh\_t4        	&	       National threshold for lower house on the fourth tier   	&	Numeric	\tabularnewline\addlinespace
lhelc\_dtrsh\_t4        	&	       District threshold for lower house on fourth tier       	&	Numeric	\tabularnewline\addlinespace
lhelc\_bon\_sts 	&	       Majority seat bonus     	&	Numeric	\tabularnewline\addlinespace
lhelc\_esys\_cmt        	&	       Comment on electoral system     	&	Text	\tabularnewline\addlinespace
lhelc\_cmt      	&	       Comments on lower house elections       	&	Text	\tabularnewline\addlinespace
lhelc\_esys\_src        	&	       Source of inforamtion on electoral system       	&	Text	\tabularnewline\addlinespace
lhelc\_lsq      	&	       Gallagher's Least-square index (LSq) of disproportionality\footnote{\label{GallgherLSQ}\citet*{Gallagher1991,Gallagher1992}} 	&	Numeric	\tabularnewline\addlinespace
lhelc\_vola\_sts        	&	       Seat A volatility\footnote{\label{a_volatitlity}Volatility arising from new entering and retiering parties, respectively \citep{Powell&Tucker2013}.}   	&	Numeric	\tabularnewline\addlinespace
lhelc\_volb\_sts        	&	       Seat B volatility\footnote{\label{b_volatitlity}Volatility arising from gaines and losses of stable parties \citep{Powell&Tucker2013}.} 	&	Numeric	\tabularnewline\addlinespace
lhelc\_vola\_vts        	&	       Vote A volatility\footref{a_volatitlity}       	&	Numeric	\tabularnewline\addlinespace
lhelc\_volb\_vts        	&	       Vote B volatility\footref{b_volatitlity}       	&	Numeric	\tabularnewline\addlinespace
lhelc\_src              	&	 Sources of information on lower house elections 	&	Text	\tabularnewline\addlinespace
lhelc\_valid\_date & Indicates whether lower house election date has been double-checked & Boolean \tabularnewline\addlinespace
\end{longtable}
\end{center}%ing
