\section{Upper House}\label{sec_upper_house}
This table provides basic information on upper houses, including start date of legislature and the total number of seats. 
Rows are compositions of upper houses. 
A new upper house composition is included when
\begin{itemize}\itemsep-4pt \parsep0pt
\item[a)]the composition changes through legislative elections, or
\item[b)]mergers or splits in factions occur during the legislature.
\end{itemize} 
Obviously, information is only provided for countries with bicameral systems.

\subparagraph{Upper house start date}
PCDB codes the date of the first meeting in the first legislative session of a new upper house as its start date. If no information on these events was available, the default is equal to the corresponding election date. 
%The table contains the following variables:

\begin{center}
\begin{longtable}{L{4cm} L{6cm} L{3cm}}
\caption{Variables in Upper House\label{tab_upper_house}}
\tabularnewline\addlinespace

%& &\tabularnewline
\toprule\toprule
\emph{\textbf{Variable}}        &       \emph{\textbf{Description}}     &       \emph{\textbf{Format}}  \tabularnewline
\midrule
\endfirsthead

\caption[]{\emph{\ldots\ continued}}\tabularnewline\addlinespace
%& &\tabularnewline
\toprule\toprule
\emph{\textbf{Variable}}        &       \emph{\textbf{Description}}     &       \emph{\textbf{Format}}  \tabularnewline
\midrule
\endhead



\addlinespace
\multicolumn{3}{c}{{\emph{continued on next page \ldots}}} \tabularnewline\addlinespace%\bottomrule
\endfoot

\bottomrule\bottomrule
\endlastfoot
uh\_id  	&	       Upper house identifier  	&	Numeric(5,0)	\tabularnewline\addlinespace
uh\_prv\_id     	&	       Identifier of previous upper house      	&	Numeric(5,0)	\tabularnewline\addlinespace
uhelc\_id       	&	       Upper house election identifier 	&	Numeric(5,0)	\tabularnewline\addlinespace
ctr\_id 	&	       Country identifier      	&	Integer	\tabularnewline\addlinespace
uh\_sdate       	&	       Upper house start date  	&	YYYY-MM-DD	\tabularnewline\addlinespace
uh\_sts\_ttl    	&	       Total number of seats in the upper house        	&	Numeric	\tabularnewline\addlinespace
uh\_cmt 	&	      Comments        	&	Text	\tabularnewline\addlinespace
uh\_src 	&	       Sources of information on upper house   	&	Text	\tabularnewline\addlinespace
uh\_valid\_sdate & Indicates whether upper house start date has been double-checked & Boolean \tabularnewline\addlinespace
\end{longtable}
\end{center}
