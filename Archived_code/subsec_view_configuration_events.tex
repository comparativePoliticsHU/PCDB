\subsection{Configuration Events}\label{subsec_view_configuration_events}
View \texttt{view\_configuration\_events} is based on tables Cabinet, Lower House, Upper, House, Presidential Elections and Veto Points, and provides the primary information on political configurations, namely country identifiers, a political configurations' start date, and the identifier values (IDs) of corresponding institutional configurations.

Accordingly, every row corresponds to a historically unique political configuration of a country's government, lower house, upper house, the position of the Head of State, and the veto institutions in place. 
, and 
%a configuration is uniquely identified by combinations of \texttt{ctr\_id}, \texttt{cab\_id}, \texttt{lh\_id}, \texttt{uh\_id} (if applies), and \texttt{prs\_id} (if applies).
because configuration start dates are identical with the start date of the institution the most recent change  occured, political configurations are uniquely identified by combinations of \texttt{ctr\_id} and \texttt{sdate}).

View \texttt{view\_configuration\_events} thus sequences changes in the political-institutional configurations of a country by date.
A new political configuration is recorded when one of the following changes occurs at one point in time during the respective period of coverage of a given country:
\begin{itemize}\itemsep-4pt \parsep0pt
\item[-]{A change in cabinet composition (rows in table Cabinet, identified by \texttt{cab\_id} or unique combinations of \texttt{cab\_sdate} and \texttt{ctr\_id}).}
\item[-]{A change in lower house composition (rows in rable Lower House, identified by \texttt{lh\_id} or unique combinations of \texttt{lh\_sdate} and \texttt{ctr\_id}).}
\item[-]{If exists in the respective country, a change in upper house composition (rows in table Upper House, identified by \texttt{uh\_id} or unique combination of \texttt{uh\_sdate} and \texttt{ctr\_id}).}
\item[-]{If exists in the respective country, a change in presidency (rows in table Presidential Election, identified by \texttt{prselc\_id} or unique combination of \texttt{prs\_sdate} and \texttt{ctr\_id}).}
\item[-]{A change in the veto power of an instituion (rows in table Veto Poinst, identified by \texttt{vto\_inst\_id} or unique combination of \texttt{ctr\_id}, \texttt{vto\_inst\_typ} and \texttt{vto\inst\_sdate}).}
\end{itemize}

Hence, changes in political configurations are either due to a change in the partisan composition of some institution, i.e., a change in the (veto-)power relations \emph{within} the institution, and consquently reflect changes in the (veto-)power relations \emph{between} the institutions.\footnote{Cases where \ldots constitute exceptions.}
Or a new configuration is recorded due to party splits or merges, newly elected upper or lower houses, or new presidencies, that not necessarly affect the respective instituional veto potential vis\`a-vis the government

View \texttt{view\_configuration\_events} is programmed as follows:

\lstinputlisting[%caption={Code to compile configuration events.},
language=SQL]%
{../SQL-codes/view_configuration_events.sql}

Rows are reported for all temporarily corresponding combinations of institutional configurations. 
Table \ref{tab_example1_australian_configs} 1 illustrates this for the Australian case.
Note that the very first configuraton of each country regularly has a non-trivial missings.
%Because one institutional configuration usually has an earlier start date than others
%(cabinets, for instance, are formed from lower houses compositions; hence, a new cabinet usually starts only after a new lower hosue is formed),
%this first configuration event has a null.
In the Australian case, the first recorded Australian cabinet started on November 1st, 1946; thus, no corresponding cabinet can be assigned to the first recorded lower house and upper house configuration (first row of Table \ref{tab_example1_australian_configs}). 
This makes it impossible to determine veto constellations for the very first recorded configuration event, resulting in missing information.

\begin{table}[h!]
\centering\footnotesize
\caption{Example of first Australian configurations.}
\label{tab_example1_australian_configs}
\begin{tabular}{r r r r r r}
\tabularnewline\toprule
\multicolumn{1}{r}{\texttt{\smallfont ctr\_id}}	&
\multicolumn{1}{r}{\texttt{\smallfont sdate}}	&	
\multicolumn{1}{r}{\texttt{\smallfont prselc\_id}}	&
\multicolumn{1}{r}{\texttt{\smallfont uh\_id}}	&
\multicolumn{1}{r}{\texttt{\smallfont lh\_id}}	&	
\multicolumn{1}{r}{\texttt{\smallfont cab\_id}}	\\\midrule 
1	& 1946-09-28	&	& 1001	& 1001	& 		\\
1	& 1946-11-01	& 	& 		& 		& 1001 	\\
1	& 1947-07-01	&	& 1002	&		&		\\\bottomrule
\end{tabular}
\end{table}

From the conceptional point of view, these incomplete configurations generally provide no information on the institutional-political setting of legislation. 
In order to provide an overview over countries' political history, these `incomplete configurations' are reported, however.

The Configuration Events View is used to create an equivalent {\em materialized} view (see section \ref{sec_mview_configuration_events}), which is, in turn, used to trigger-in configuration end dates (see subsection \ref{trg_mv_config_ev_edate}) and to identify temporarily corresponding institutional configurations (see subsection \ref{trg_mv_config_ev_corresponding_ids}).