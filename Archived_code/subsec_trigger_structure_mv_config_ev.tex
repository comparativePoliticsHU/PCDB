\subsection{Configuration Events Materialized View trigger structure}\label{subsec_trigger_structure_mv_config_ev}

Whenever a change on the base tables Cabinet, Lower House, Upper House, Presidential Elections, and Veto Points occurs, the Configuration Events View is up-to-date when queried; the Configuration Events Materialized View, due to its `eagerness' is not, though.
A number of triggers defined on the base tables and two functions guarantee that a change on a base table is propagated trhought ot the materialized view Configuration events; this structure is illustrated in Figure \ref{fig_mv_config_ev_triggers}.

\input{./flow_charts/flow_chart_template}\label{fig_mv_config_ev_triggers}

Central to the structure implemented to update configuration events displayed in Figure \ref{fig_mv_config_ev_triggers} are two function, which will be described in turn.

\subsubsection{Function \texttt{mv\_config\_ev\_refresh\_row()}}

A change on a base table triggers a refresh of affected rows in the Configuration Events Materialised View:
\begin{itemize}

\item[-]{On update of columns having the institutional configuration identifier or start date values listed in the materialized view, function \texttt{mv\_config\_ev\_*\_ut()} is called, where the asterisk \texttt{*} is a placeholder for the table name.
This function will perform one call of function \texttt{mv\_config\_ev\_refresh\_row()} with old country identifier and start date values (note that start date refers to the configuration start date at the level of the base table, e.g. \texttt{cab\_sdate} or \texttt{prs\_sdate}), and another call with new (i.e., updated) country identifier and configuration start date values for each row that is updated.}

\item[-]{On insert into a base table function \texttt{mv\_config\_ev\_*\_it()} is called, which performs a call of \texttt{mv\_config\_ev\_refresh\_row()} with newly inserted country identifier and configuration start date values for each row that is inserted.}

\item[-]{On delete from a base table call function \texttt{mv\_config\_ev\_*\_dt()} is calles, which performs a call of \texttt{mv\_config\_ev\_refresh\_row()} with the country identifier and start date values of the row that is removed for each row that is deleted.}
\end{itemize}

These event triggers are defined on each of the base tables and named \texttt{mv\_config\_ev\_update}, \texttt{mv\_config\_ev\_insert}, and \texttt{mv\_config\_ev\_delete}, respectively. 

Function \texttt{mv\_config\_ev\_refresh\_row()} performs a refresh of rows in materialized view Configuration Events for a given combination of country identifier and start date.
It executes the following actions:
\begin{itemize}
\item[(i)]{It disables all triggers implemented on materialized view Configuration Events;}
\item[(ii)]{deletes the row from materialized view Configuration Events that is identified by input arguments country identifier and start date (\texttt{ctr\_id} and \texttt{sdate});}
\item[(iii)]{inserts the respective configuration information (country identifier and start date) from {\em view} Configuration Events into {\em materialized view} Configuration Events;}
\item[(iv)]{enables all triggers implemented on materialized view Configuration Events;}
\item[(v)]{updates all columns containing the affected institution identifiers in order to trigger \texttt{trg\_mv\_config\_ev\_correspond\_ids}; and}
\item[(vi)]{updates column containing configuration end dates (\texttt{edate}) of the configurations of the same country that have a younger start date younger than the currently refreshed row (for odler start and end dates will not be affected by refresh).}
\end{itemize}

The function is defined as follows:
\lstinputlisting[%caption={Code to create row-refresh function on materialized view Configuration Events.},
language=postgreSQL]%
{../SQL-codes/functions/fun_mv_config_ev_refresh_row.sql}

\subsubsection{Functions \texttt{mv\_config\_ev\_\#\_id\_*\_trg}}

Because \texttt{mv\_config\_ev\_refresh\_row()} only affects rows in materialised view Configuration Events identified by input arguments country identifier and start date, not all rows in which an institution-configuration ID is listed will be affected (recall that one institutional configuration may correspond to multiple configuration events).
Hence, a change in a base table that affects the configuration identifier of this institutional configuration requires to propagate this change through all configuration events in the materialized vies that are associated with this identifier. 

This is achieved by a set of triggers named \texttt{mv\_config\_ev\_\#\_id\_*\_trg}, where the hastag stands for the institutions (i.e., is \texttt{cab}, \texttt{lh}, \texttt{uh}, \texttt{lhelc}, or \texttt{prselc}), and the asterisk is a placeholder for trigger events update (\texttt{ut}), insert (\texttt{it}), or delete (\texttt{dt}):
\begin{itemize}
\item[-]Trigger \texttt{mv\_config\_ev\_\#\_id\_ut\_trg} calls function \texttt{mv\_config\_ev\_\#\_id\_ut\_trg()} on update of the identifier column, which performs function \texttt{mv\_config\_ev\_ut\_\#\_id()} with the two input arguments old and new identifier.
\texttt{mv\_config\_ev\_ut\_\#\_id()} updates materialized view Configuration Events and sets all identifier values to the new identifier value where they are currently equal to the old identifier value.

\item[-]Trigger \texttt{mv\_config\_ev\_\#\_id\_it\_trg} calls function \texttt{mv\_config\_ev\_\#\_id\_it\_trg()}, which executes an update of materialized view Configuration Events, setting the respective identifier column equal to its actually values, which will trigger the inserting of corresponding IDs (implemented by yet another trigger defined on materialised view configuration events)

\item[-]Trigger \texttt{mv\_config\_ev\_\#\_id\_dt\_trg} calls function \texttt{mv\_config\_ev\_\#\_id\_dt\_trg()} on delete of a row in the respective base table, which performs function \texttt{mv\_config\_ev\_dt\_*\_id()} with the old (i.e., to-be-removed) identifier value as single input argument.
\texttt{mv\_config\_ev\_dt\_\#\_id()} updates materialized view Configuration Events and sets all identifier values to \texttt{NULL} where they are equal to the old identifier value.
\end{itemize}

